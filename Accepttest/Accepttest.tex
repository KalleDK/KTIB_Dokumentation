%!TEX root = ../main.tex

\chapter{Accepttest}

\begin{table}[H]
\centering
{\rowcolors{2}{white!80!black!30}{white!70!black!60} %farver på hver anden række -starter på 3
\setlength{\arrayrulewidth}{0.2mm}					 %tykkelse på linier 
\setlength{\tabcolsep}{10pt}						 %indryk i celle 
\renewcommand{\arraystretch}{1.5}					 %højden på tabelrum
\center
\begin{tabular}{|p{4cm}|p{4cm}|p{4cm}|}		 %længden på alle rum
\hline
\multicolumn{3}{|>{\columncolor{white!20!black!90}}m{13.44cm}|}{\textcolor{white}{\large{\textbf{Revision}}}} \\\hline
\rowcolor{white!70!black!60}
\textcolor{black}{\large{\textbf{Ændret af}}}&
\textcolor{black}{\large{\textbf{Version}}}&	
\textcolor{black}{\large{\textbf{Dato}}}\\
\hline
Alle	& 1	 	& 23-02-2015  \\
		& 		&   \\
		& 		&   \\
		& 	 	&   \\
\hline
\end{tabular}
}
\caption{Revision for accepttest}
\label{table:RevAccept}
\end{table}

\section{Test setup}
Til at teste følgende skal der bruges et setup med en \gls{pc} der er i stand til at forbinde til den indlejrede Linux platform. \glslink{I2CSensor}{I2CSensorer} tilsluttes til den indlejrede Linux platform via de PSoC moduler der styre dem. Aktuatorerne skal ligeledes tilsluttes gennem deres respektive moduler.
  
\section{Accepttests}
%Accepttest til Usecase1
%!TEX root = ../main.tex

\accepttest[UC1]{Use Case 1 Aflæs målinger}{
Bruger tilgår \gls{gui} ved at indtaste \gls{mgmturl} i sin webbrowser	 		& Systemet viser \gls{gui} 	&  	& \\
Bruger trykker på et oprettet kar i \gls{gui}en	 								& Systemet viser et skærmbillede med oversigt over karet	&  	& \\
Bruger aflæser pH-værdien under kar	data ved målt værdi					& En værdi mellem 0-14 aflæses ved målt værdi	&  	& \\
Bruger aflæser vandniveau under kar	data ved målt værdi					& En værdi med enheden liter aflæses ved målt værdi 	&  	& \\
Bruger aflæser jordfugtighed under kar data ved målt værdi				& En værdi med enheden $\%$ aflæses ved målt data & & \\
}

%Accepttest til Usecase2
%!TEX root = ../main.tex

\accepttest[UC2]{Use Case 2 Manuel vanding}{
Bruger tilgår \gls{gui} ved at indtaste \gls{mgmturl} i sin webbrowser 	& Systemet viser \gls{gui}	 								&  	& \\
Bruger trykker på et oprettet kar i \gls{gui}en	 						& Systemet viser et skærmbillede med oversigt over karet hvor man kan tilgå manuel vanding 	&  	& \\
Bruger trykker på manuel vanding			 	& Systemet begynder at vande 	&  	& \\

Bruger trykker på stop manuel vanding		 	& Systemet stopper med at vande  &  	& \\
}

%Accepttest til Usecase3
%!TEX root = ../main.tex

\accepttest[UC3]{Use Case 3 Indtast pH-værdi}{
Bruger tilgår \gls{gui} ved at indtaste \gls{mgmturl} i sin webbrowser	 		& Systemet viser \gls{gui} 	&  	& \\
Bruger trykker på et oprettet kar i \gls{gui}en	 							& Systemet viser et skærmbillede med oversigt over karet hvor man kan tilgå indtastning af pH-værdien 	& & \\
Bruger trykker på feltet uden for "pH-Værdi", retter værdien til en pH-værdi på 7.5 og trykke "Gem data"	& Systemet opdaterer pH-værdien til  7.5		& 	& \\
}

%Accepttest til Usecase4
%!TEX root = ../main.tex

\accepttest[UC3]{Use Case 4 Karstyring}{
Tryk på Karstyring på interfacet	& Der forekommer 2 valmuligheder &  & \\
Tryk på PH-værdi   					& Der gives mulighed for at indtaste data	& 	& \\
Indtast 7 og tryk ok 				& PH-værdien opdateres til 7   &	& \\
Tryk på OK   						& Menuen returnerer   &  	& \\
Tryk på Volumen 					& Der gives mulighed for at indtaste data		& 		& \\
Indtast 100 						& Volumen opdateres til 100L 		&    & \\
Tryk på OK 		    				& Menuen returnerer	        &    & \\
Tryk på OK 							& Cirkulations pumpe og pumperne til dosering af gødningen starter	&	  	& \\
}