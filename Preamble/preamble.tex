
%%%%%%%%%%%%%%%%%%%%%%%Opsætning af format%%%%%%%%%%%%%%%%%%%%%%%%%%
\documentclass[a4paper,oneside]{memoir} %A4papir, to side, størrelse 12, type memoir 

% For dansk opsætning med æ, ø og å, samt pænere orddeling.
\usepackage[utf8]{inputenc}				% æøå
\usepackage[danish]{babel}				% dansk opsætning
\renewcommand{\danishhyphenmins}{22}	% fikser babel fejl/bedre orddeling
\usepackage[T1]{fontenc}
\usepackage{lmodern} 

%%Høre til under tabel (Pakker, men skal stå før pgfplot pga. default værdier)%%
\usepackage[table]{xcolor}				
%%%%%%%%%%%%%%%%%%%%%%%%%%%%%%%%%%%%%%%%%%%%%%%%%%%%%%%%%%%%%%%%%%%%%%%%%%%%%%%%

\setcounter{tocdepth}{4} % inkludere sub + subsubsection i inholdsfortegnelde
\setsecnumdepth{subsection}

\usepackage{titlesec} 
\titleformat{\chapter}		  % Fjerne Kapitel og tal fra chapters
			{\Large\bfseries} % format
			{}                % label
			{0pt}             % sep
			{\huge}           % before-code
% til mellemrum

\titlespacing\section{0pt}
{18pt plus 4pt minus 2pt}{6pt plus 2pt minus 2pt} %halvt mellemrum efter section
\titlespacing\subsection{0pt}
{12pt plus 4pt minus 2pt}{2pt plus 2pt minus 2pt} %kun lidt mellemrum efter subsection
\titlespacing\subsubsection{0pt}
{12pt plus 4pt minus 2pt}{2pt plus 2pt minus 2pt} %kun lidt mellemrum efter subsubsection 

\setlength\parindent{0pt} %Ingen indryk efter ny afsnit

%Marginer indstilles
\setlrmarginsandblock{3cm}{3cm}{*}		%Højre - venstre
\setulmarginsandblock{3cm}{2.5cm}{*}	%Øverst - nederst
\checkandfixthelayout[nearest]    		%Specifikt valg af højde algoritme
\usepackage{ragged2e,anyfontsize}		% Justering af elementer
\usepackage{fixltx2e}					% Retter forskellige fejl i LaTeX-kernen

% Sidehoved og -fod
\let\footruleskip\undefined  %fixer memoir default footruleskip
\usepackage{fancyhdr}
\pagestyle{fancy}
\fancyhf{}
\fancyhead[C]{\textit{Automatisk Vandingssystem}}
\fancyfoot[LO,RE]{\thepage\ af \thelastpage}  	% sættet sidetal tal h/v efter om der er ulige
												% eller lige sidetal
\fancypagestyle{plain}{% bruges ved Undtagelser						
  \fancyhf{}%
  \fancyfoot[RO,LE]{\thepage\ af \thelastpage}	
  \renewcommand{\headrulewidth}{0pt}			%Ingen linje ved chapter, kun sidetal
}
%%%%%%%%%%%%%%%%%%%%%%%%%%%%%%%%%%%%%%%%%%%%%%%%%%%%%%%%%%%%%%%%%%%

%%%%%%%%%%%%%%%%%%%%Pakker og design til forside%%%%%%%%%%%%%%%%%%%
\usepackage{amsmath}
\usepackage{tikz}
\usepackage{epigraph}


\renewcommand\epigraphflush{flushright}
\renewcommand\epigraphsize{\normalsize}
\setlength\epigraphwidth{1\textwidth}

\definecolor{titlepagecolor}{cmyk}{1,.60,0,.40}

\DeclareFixedFont{\titlefont}{T1}{ppl}{b}{it}{0.5in}




% The following code is borrowed from: http://tex.stackexchange.com/a/86310/10898

\newcommand\titlepagedecoration{%
\begin{tikzpicture}[remember picture,overlay,shorten >= -10pt]

\coordinate (aux1) at ([yshift=-15pt]current page.north east);
\coordinate (aux2) at ([yshift=-410pt]current page.north east);
\coordinate (aux3) at ([xshift=-4.5cm]current page.north east);
\coordinate (aux4) at ([yshift=-150pt]current page.north east);

\begin{scope}[titlepagecolor!40,line width=12pt,rounded corners=12pt]
\draw
  (aux1) -- coordinate (a)
  ++(225:5) --
  ++(-45:5.1) coordinate (b);
\draw[shorten <= -10pt]
  (aux3) --
  (a) --
  (aux1);
\draw[opacity=0.6,titlepagecolor,shorten <= -10pt]
  (b) --
  ++(225:2.2) --
  ++(-45:2.2);
\end{scope}
\draw[titlepagecolor,line width=8pt,rounded corners=8pt,shorten <= -10pt]
  (aux4) --
  ++(225:0.8) --
  ++(-45:0.8);
\begin{scope}[titlepagecolor!70,line width=6pt,rounded corners=8pt]
\draw[shorten <= -10pt]
  (aux2) --
  ++(225:3) coordinate[pos=0.45] (c) --
  ++(-45:3.1);
\draw
  (aux2) --
  (c) --
  ++(135:2.5) --
  ++(45:2.5) --
  ++(-45:2.5) coordinate[pos=0.3] (d);   
\draw 
  (d) -- +(45:1);
\end{scope}
\end{tikzpicture} }
%%%%%%%%%%%%%%%%%%%%%%%%%%%%%%%%%%%%%%%%%%%%%%%%%%%%%%%%%%%%%%%%%%%%

%%%%%%%%%%%%%%%%%%%%%%%%%%%%%%Pakker%%%%%%%%%%%%%%%%%%%%%%%%%%%%%%%%
\usepackage{graphicx} 				% Haandtering af eksterne billeder (JPG, PNG, EPS, PDF)
\usepackage{multirow}               % Fletning af raekker og kolonner (\multicolumn og \multirow)

\usepackage{wrapfig}				%for figure der skal have tekst om sig
\usepackage{float}					% Muliggoer eksakt placering af floats, f.eks. \begin{figure}[H]

\usepackage{enumerate}				% Muliggoer at man kan bruge fx a) i enumerate

\usepackage{pdflscape}				% Muligør landskab på enkelte sider
\usepackage{tabularx}				% Tabeller med X width

\usepackage{Usecases/usecases}				% Usecase

\usepackage{mathtools}				% Formler og matematik

\usepackage{caption}
%\usepackage{subcaption}			% This package kills me :/

\usepackage{pdfpages}				% til at inkludere PDF filer

\usepackage{gensymb}				% symboler til latex

% For at holde styr på mangler i teksten
\usepackage[footnote,draft,danish,silent,nomargin]{fixme}	
%%%%%%%%%%%%%%%%%%%%%%%%%%%%%%%%%%%%%%%%%%%%%%%%%%%%%%%%%%%%%%%%%%%%

% If then
\usepackage{ifthen}

% ¤¤ Litteraturlisten ¤¤ %
\usepackage[danish]{varioref}				% Muliggoer bl.a. krydshenvisninger med sidetal (\vref)
\usepackage{natbib}							% Udvidelse med naturvidenskabelige citationsmodeller
\bibpunct[,]{[}{]}{;}{a}{,}{,} 				% Definerer de 6 parametre ved Harvard henvisning 
											% (bl.a. parantestype og seperatortegn)
\bibliographystyle{Litteratur/harvard}		% Udseende af litteraturlisten.
\usepackage{hyperref}

% Ordliste
\usepackage[nonumberlist,toc]{glossaries}
\makeglossaries
\newglossaryentry{plante}{
	name=plante,
	description={ består af flora og gromedie}
}

\newglossaryentry{gromedie}{
	name=gromedie,
	description={ er det stof floran er plantet i, dette kan fx være muld}
}

\newglossaryentry{mgmturl}{
	name=management url,
	description={ adressen hvorpå guiinterfacet befinder sig}
}

\newglossaryentry{pc}{
	name=PC,
	description={ computer med Windows 7+ styresystem, samt Google Chrome som browser}
}

\newglossaryentry{I2CSensor}{
	name=I2CSensor,
	description={ er en samlet generisk beskrivelse af måleinstrumenter, der kan tilsluttes en Sensor Ø}
}

\newglossaryentry{kar}{
    name=kar,
    description={ er en beholder, der kan indeholde gødningsmix }
}

\newglossaryentry{goedningsmix}{
    name=gødningsmix,
    description={ er en blanding af vand og gødning med en bestemt pH-værdi }
}

\newglossaryentry{gui}{
	name=gui,
	description={ er den grafiske brugergrænseflade}
}

\newglossaryentry{flexpms}{
    name=FlexPMS,
    description={ (Flexible Plant Management System) er den software, som binder brugergrænsefladen med den fysiske verdens}
}

\newglossaryentry{database}{
    name=database,
    description={ gemmer brugerens indstillinger samt log}
}

\newglossaryentry{centralcontroller}{
    name=CentralController,
    description={ er systemets centrale computer}
}

\newglossaryentry{kargruppe}{
    name=KarGruppe,
    description={ er et vandkar med tilhørende karkontroller og sensor ø'er}
}

\newglossaryentry{karcontroller}{
    name=KarController,
    description={ styrer tilgangen af vand, gødning og pH-væske samt de sensor ø'er der er tilkoblet denne}
}

\newglossaryentry{rsconverter}{
    name=RSConverter,
    description={ et elektronisk print, som kan konvertere mellem UART 232 og RS485}
}

\newglossaryentry{ventilstyring}{
    name=VentilStyring,
    description={ en ventil, der kan åbnes og lukkes vha. et 5V-signal. Ventilen er tilsluttet 12V }
}

\newglossaryentry{sensoroe}{
    name=Sensor Ø,
    description={ består af en Sensor Ø Controller, en række I2CSensorer, som måler fra et begrænset området, og en Doseringsventil }
}

\newglossaryentry{sensoroecontroller}{
    name=Sensor Ø Controller,
    description={ er controlleren i en Sensor Ø, som opsamler data fra sensorerne og kan styre Doseringsventilen }
}

\newglossaryentry{doseringsventil}{
    name=doseringsventil,
    description={ åbner og lukker for tilførslen af gødningsmix til en bestemt Sensor Ø }
}

\newglossaryentry{flowmaaler}{
    name=flowmåler,
    description={ måler mængden af vand som løber gennem denne}
}






\newcommand{\systemBDD}[3][FLAF]{
\ifthenelse{ \equal{#1}{FLAF} }{
\begin{figure}[H]
	\centering
	\includegraphics[width=#2\textwidth]{Systemarkitektur/#3/#3_BDD.png}
	\label{fig:#3_BDD}
	\caption{Block Definition Diagram af #3}
\end{figure}
}{
\begin{figure}[H]
	\centering
	\includegraphics[width=#2\textwidth]{Systemarkitektur/#1/#1_BDD.png}
	\label{fig:#1_BDD}
	\caption{Block Definition Diagram af #3}
\end{figure}
}

}

\newcommand{\systemIBD}[3]{
\begin{figure}[H]
	\centering
	\includegraphics[width=#1\textwidth]{Systemarkitektur/#2/#2_IBD.png}
	\label{fig:#2_IBD}
	\caption{Internal Block Diagram af #3}
\end{figure}
}

\newcommand{\systemDomainModel}[3]{
\begin{figure}[H]
	\centering
	\includegraphics[width=#1\textwidth]{Systemarkitektur/#2/#2_Domain_Model.png}
	\label{fig:#2_Domain_Model}
	\caption{Domænemodel af #3}
\end{figure}
}

\newcommand{\systemAllokeringsDiagram}[3]{
\begin{figure}[H]
	\centering
	\includegraphics[width=#1\textwidth]{Systemarkitektur/#2/#2_AllokeringsDiagram.png}
	\label{fig:#2_Allokeringsdiagram}
	\caption{Allokeringsdiagram af #3}
\end{figure}
}

\newcommand{\RevisionsTabel}[2]
{
\begin{table}[H]
\centering
{\rowcolors{2}{white!80!black!30}{white!70!black!60} %farver på hver anden række -starter på 3
\setlength{\arrayrulewidth}{0.2mm}					 %tykkelse på linier 
\setlength{\tabcolsep}{10pt}						 %indryk i celle 
\renewcommand{\arraystretch}{1.5}					 %højden på tabelrum
\center
\begin{tabular}{|p{4cm}|p{4cm}|p{4cm}|}		 %længden på alle rum
\hline

\multicolumn{3}{|>{\columncolor{white!20!black!90}}m{13.44cm}|}{\textcolor{white}{\large{\textbf{Revision}}}} \\\hline
\rowcolor{white!70!black!60}
\textcolor{black}{\large{\textbf{Ændret af}}}&
\textcolor{black}{\large{\textbf{Version}}}&	
\textcolor{black}{\large{\textbf{Dato}}}\\
\hline
#2
\hline
\end{tabular}
}
\caption{Revision for #1}
\label{table:#1_Revision}
\end{table}
}

\newcommand{\systemSignaler}[3][FLAF]{
\begin{table}[H]
\centering
{\rowcolors{2}{white!80!black!30}{white!70!black!60} %farver på hver anden række -starter på 3
\setlength{\arrayrulewidth}{0.2mm}					 %tykkelse på linier 
\setlength{\tabcolsep}{10pt}						 %indryk i celle 
\renewcommand{\arraystretch}{1.5}					 %højden på tabelrum
\center
\begin{tabular}{|p{20mm}|p{40mm}|p{30mm}|p{30mm}|}		 %længden på alle rum
\hline

\multicolumn{4}{|>{\columncolor{white!20!black!90}}m{14.20cm}|}{\textcolor{white}{\large{\textbf{Signal beskrivelser}}}} \\\hline
\rowcolor{white!70!black!60}
\textcolor{black}{\large{\textbf{Navn}}}&
\textcolor{black}{\large{\textbf{Definition}}}&	
\textcolor{black}{\large{\textbf{Område}}}&
\textcolor{black}{\large{\textbf{Kommentar}}}\\
\hline
#3
\hline
\end{tabular}
}
\ifthenelse{ \equal{#1}{FLAF} }{
\caption{Signal beskrivelser for #2}
\label{sig:#2_Signal}
}{
\caption{Signal beskrivelser for #2}
\label{sig:#1_Signal}
}

\end{table}
}

\newcommand{\accepttest}[3][FLAF]
{
\begin{table}[H]
\centering
{\rowcolors{2}{white!80!black!30}{white!70!black!60} %farver på hver anden række -starter på 3
\setlength{\arrayrulewidth}{0.2mm}					 %tykkelse på linier 
\setlength{\tabcolsep}{10pt}						 %indryk i celle 
\renewcommand{\arraystretch}{1.5}					 %højden på tabelrum
\center
\small
\begin{tabular}{|p{4cm}|p{4cm}|p{3cm}|p{2.5cm}|}		 %længden på alle rum
\hline

\multicolumn{4}{|>{\columncolor{white!20!black!90}}m{15.67cm}|}{\textcolor{white}{\large{\textbf{Accepttest #2}}}} \\\hline
\rowcolor{white!70!black!60}
\textcolor{black}{\large{\textbf{Test}}}&
\textcolor{black}{\large{\textbf{Forventet resultat}}}&	
\textcolor{black}{\large{\textbf{Resultat}}}&
\textcolor{black}{\large{\textbf{Godkendt Kommentar}}}\\
\hline
#3
\hline
\end{tabular}
}
\ifthenelse{ \equal{#1}{FLAF} }{
\caption{Acceptest for #2}
\label{table:#2_AcceptTest}
}{
\caption{Acceptest for #2}
\label{table:#1_AcceptTest}
}
\end{table}
}