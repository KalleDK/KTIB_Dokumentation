\section{AVSLibrary}

AVSLibrary'et er en samling af custom Psoc4 komponenter, 
der gør det lettere at overskue og udvikle selve Kar, Sensor Ø og Fieldsensor. 
Der er mange ting der bliver lettere, hvis ventilerne skal inverte signalet, 
er det et enkelt sted dette skal ændres, og så bliver det rettet i alle andre 
projekter hvor de bruges. Der er også begrænsninger med hensynt til at have 
shared kode. Enums og Struct er fælles på tværs af komponenterne, for at 
sikre at Enums har de rigtige værdier og Structs er lettere at håndtere et 
samlet sted. Fieldsensor er blevet meget strømlinet via dette, da man 
blot indsætter en SensorBus, konfigurerer I2C adresse og sætter den til 
at være Fieldsensor. Derefter skal mainloopet blot opdateret high og low 
værdierne, samt kalde Communicate funktionen. Skulle I2C delen udskiftes 
med en anden bus ville det kunne gøres uden at påvirke de enkelte Fieldsensors. 
De vil blot få den nye bus med næste gang den kompileres.
Selve komponenterne er skrevet i noget "meta" c kode, da der er nogle 
specielle Psoc macro koder der bruges til at generere den færdige c kode 
for komponenten. Hvor den mest brugte er `\$INSTANCE\_NAME` som erstattes 
af det navn brugeren vælger for komponenten, derfor bør alle metoder og variabler 
prefikses med dette. Dette sikrer at alle metoder og variabler er unikke, når 
c koden genereres af brugeren.

\subsection{DebugUart}
\subsubsection{DebugUart}

\subsection{Doseringspumpe}
\subsubsection{Formål}
Formålet med denne komponent er at lave en ens tilgang 
til pumper i systemet. Komponenten består af et Clock 
input, en PWM generator, samt en Pin. Clocken er input 
til PWM generatoren, som outputter sit signal til Pin'en. 
Dette signal bliver derefter brugt på et pumpe 
styreboard. Selve PWM signalet er et Fast PWM signal 
hvor vi sætter comparatoren efter de procenter den 
bliver indstillet til. Der kan på komponenten vælges om 
signalet skal invertes. Hastigheden af PWM signalet 
angives i procent. Komponenten har 2 states running (0x01) 
og blocked (0x00). Dette angiver om PWM generatoren er 
indstillet til at sende det ønskede signal. 
Man kan derved instille en hastighed inden man starter 
pumpen, skulle dette være ønsket, samt huske pumpen den 
sidste værdi den var indstillet til, hvis man midlertidig 
vil stoppe pumpen og starte den igen med samme hastighed.

\subsubsection{Funktioner}
Her ses funktionerne der er tilknyttet komponenten.

\funk{void Pumpe\_Start(void)}
{Funktionen bruges til at initialisere PWM generatoren}
{Void}
{}

\funk{void Pumpe\_Run(void)}
{Indstiller PWM generatoren med den forprogrammerede hastighed, hvis den tidligere hastighed var 0\%, bliver hastigheden sat til 100\%. Pumpens state bliver ændret til running (0x01)}
{Void}
{}

\funk{void Pumpe\_Block(void)}
{Stopper PWN generatoren og ændre pumpens state til blocked (0x00)}
{Void}
{}

\funk{void Pumpe\_SetSpeed(uint8 percent)}
{Indstiller hastigheden PWM generatoren ønskes at køre med, er staten running (0x01) bliver PWM generatoren's comperator ændret med det samme}
{Void}
{
\funkArg{percent}{0 - 100, hvor mange procent af tiden signalet skal være aktivt.}
}

\funk{void Pumpe\_ApplySpeed(void)}
{Intern funktion der indstiller PWM Generatoren med den ønskede hastighed}
{Void}
{}

\funk{uint8 Pumpe\_GetSpeed(void)}
{Returnerer den indstillede hastighed i procent}
{0 - 100 procent}
{}

\funk{void Pumpe\_DebugHandle(const char ch)}
{Kan bl.a. starte, stoppe pumpen, samt ændre hastighed på signalet. Debug handler, se mere under DebugUart}
{Void}
{
\funkArg{ch}{Input char}
}

\funk{void Pumpe\_DebugState(void)}
{Udskriver hvilken state pumpen er i, samt hastighed mm.}
{Void}
{}

\subsection{FlowSensor}
%!TEX root = ../../main.tex

\section{Flowsensor}
Til at kontrollere vandflowet til systemet, er valgt at benytte en flow sensor af typen YF S201\footnote{\citet{nxp:YFS201}}. 
Dette er en Hall Effect sensor. En Hall Effekt sensor benytter ændringer i et nær-magnetfelt til at 
ændre sensorens outputspænding, hvilket genererer et PWM-output der identificerer den mængde vand 
der flyder igennem sensoren, og kan omregnes til mængde vand pr. enhed.

Flowraten udregnes efter formlen: 
				
\begin{figure}[H]
    \begin{align*}
       \frac{Pulse frequency [Hz]}{7.5} = flowrate[L/min]
    \end{align*}
\label{eq:PWM}
\caption{Beregning af flowrate}
\end{figure}				

Ved denne udregning svarer eks. $16Hz = 2L/min$ og $65.5HZ = 8L/min$. \newline
Denne måde at beregne flowet på giver i midlertidig problemer med implementering på KarPSoC'en, da dette ville kræve at programmet "lagde beslag" på processoren for at udregne flowet over en given tid. Da KarPSoC-programmet har til formål at køre adskillige opgaver vedrørende Karret, er det ganske enkelt ikke en mulighed at følge denne implementering.\newline
En alternativ implementering blev valgt for at imødekommende denne problematik. Ved praktisk at måle den mængde vand pr. antal PWM-cycle der flyder igennem sensoren, er det muligt at opstille en model for hvor meget vand der flyder til karret baseret på det antal PWM-cycles der modtages af programmet.\newline
Flowsensoren tilkobles via 3 medfølgende pins.

\begin{figure}[H]
	\begin{center}
		\begin{tabular}{ l l }
			 \textcolor{black}{Sort}:   & $GND(-)$ 		\\ 
			 \textcolor{yellow}{Gul}:   & $PWM output$ 	\\  
			 \textcolor{red}{Red}:    	& $VCC(+5V)$ 	\\
		\end{tabular}
	\end{center}
\caption{Pinoversigt}
\end{figure}

Output fra sensoren følger CMOS-standarden, det vil sige at den kan kobles direkte til en given inputpin på PSoC'en. Output-raten gives som udgangspunkt med en målepræcision på +/- 10%.
 
Under operation sinker sensoren 15mA, derved kan den kan trække sin forsyning direkte fra PSoC'en.

\subsection{Counter-kredsløb}
For at aflaste Kar-PSoC'en der håndtere flowsensoren, designes et eksternt counter-kredsløb. Herved minimeres det antal interrupts der trækkes på PSoC-programmet. Kredsløbet designes ved brug af en 4-bit counter (74HC393\footnote{\citet{nxp:74HC393}}) samt en 2-input AND-gate (74HC08\footnote{\citet{nxp:74HC08}}). 
Kredsløbet er koblet således at PWM-output fra flowsensoren kobles til clock-input 1 (CP1) på counter'en. Outputs fra denne (Q1, Q3) kobles til AND'gaten.
Kredsløbet ses på \ref{screenshot:counter} herunder. 

\begin{figure}[H]
	\centering
	\includegraphics[height=5cm]{../Hardware/Flow_Sensor/Screenshots/FlowSensor_Schematics_ver1}
	\caption{Counter-kredsløb}
	\label{screenshot:counter}
\end{figure}

Koblet på denne måde, går gaten "høj" når counteren rammer 10. Et logisk høj fra gaten, trigger et INT i PSoC'en. Som det første i ISR'rutinen nulstilles counter'eren via et logisk høj på inputtet MR1. Dette gøres for at forhindre at den dobbelttrigger når counteren efterfølgende rammer 14 counts. Info er hentet fra datasheet, \ref{screenshot:logicTable}

\begin{figure}[H]
	\centering
	\includegraphics[height=5cm]{../Hardware/Flow_Sensor/Screenshots/FS_logicTable}
	\caption{74HC393 logic}
	\label{screenshot:logicTable}
\end{figure}

Ved praktiske målinger forventes en $PWM_{max}=40-50Hz$.
Det giver systemet som maximum 80ms til at reset'e counteren inden en potentiel fejltriggering.

\begin{figure}[H]
    \begin{align*}
       \frac{1}{frekvens[Hz]} &= Periodetid[ms] \\
       \frac{1}{50[Hz]} &= 20[ms] \\
       4[clockcycles]*20[ms] &= 80[ms] \\ 
    \end{align*}
\label{eq:Trigger}
\caption{Beregning af trigger}
\end{figure}

Simulering af Counter-timing ses på  \ref{screenshot:counterTiming} herunder.

\begin{figure}[H]
	\centering
	\includegraphics[height=7cm]{../Hardware/Flow_Sensor/Screenshots/FlowSensor_Timingsdiagram}
	\caption{Counter-Timingsdiagram}
	\label{screenshot:counterTiming}
\end{figure}

Denne opsætning med ekstern counter, kan kun lade sig gøre fordi der ikke behøves en ultra præcis måling af vandflowet. 
Ved at lave INT hver 10 gang, midles der over vandflowet. 
Dokumentation til flowsensor-softwarestyring findes under "Software-afsnittet".


\subsection{pHProbe}
\subsubsection{pHProbe}
Softwaren til pHProbe er implementeret med en OPAmp og en ADC i PSoC 4 chippen desuden er der en interrupt service routine der bruges til at calibrere proben med. OPAmp'en og ADC er beskrevet i hardware delen af dokumentationen, i software bliver ADC'en aflæst for hvilken værdi der er i milivolt på ADC'ens indgang. Desuden er der sat default værdier for pH4 og pH7 som er disse pH proben kalibreres efter, disse to værdier bruges til at regne den aktuelle pH i forhold til det antal milivolt der læses på ADC'en.

\subsubsection{Funktioner}

\funk{CY\_ISR(pHProbe\_CALIBRATE)}
{Dette er interrupt service routinen der bruges til at kalibrere proben det er vigtigt at nævne at når denne kørers kan karret ikke kommunikere fordi at karret står og venter i denne routine}
{intet}
{
}

\funk{void pHProbe\_Start()}
{Denne funktion start alle afhængigheder pHProbe komponenten har, herunder ADC og OPAmp}
{intet}
{
}

\funk{float pHProbe\_mvToPh(float mv, float ph4, float ph7)}
{Denne funktion regner den aktuelle pHVærdi ud fra målingen fra ADCen}
{pHVærdien}
{
}

\funk{float pHProbe\_getpH()}
{Denne funktion aflæser ADC'en og returnere pHVærdien udregnet ved at kalde \textKode{pHProbe\_mvToPh} funktionen}
{antal ticks}
{
}

\funk{void FlowSensor\_calcFlowLiter()}
{Denne funktion beregner det antal liter der er talt via flow ticks}
{intet}
{
}

\subsection{RS485}
\subsubsection{Formål}
Denne komponent er en UART udbygget til at virke via en RS485 bus. 
Der er tilføjet en TX Pin der skal være høj under transmission, og 
ellers lav. UARTen er bygget via interrupts og buffer, 
hvilket sikre bedre mod tab af beskeder. TX benet er implementeret ved 
at disable RX og enable TX, tænde for TX Pin og vente 20 ms. Derefter 
sende kommandoen samt eventuelle argumenter. Når alt er sendt disables 
TX Pin, samt enables RX og TX disables. Da der ingen collison detection 
er, er protokollen lavet til at kun masteren snakker, og slaves kun 
svarer når master forventer dette. Modtagelsen af beskeder sker via interrupts 
og det er først relevant at kalde RS485\_GetRxMessage når beskeden er helt 
modtaget. Mark Space adresseringen er indbygget i UART komponenten, eller 
er modtage statemachinen ca det samme som kan ses under \ref{fig:RS485RX_StateDiagram}. 
Afsendelsen er lidt mere kompliceret, da afsendelsen af UART beskederne sker 
i hardwaren og ikke software. Derfor er vi nødt til at aktivere interrupts 
når sender bufferen er tom, og først derefter slukke for TX benet.

\StateDiagram{0.82}{AVSLibrary}{RS485TX}

\subsubsection{Funktioner}

\funk{void RS485\_Start(void)}
{Initialiserer den underlæggende UART samt 
TX Pin'en, og klargører buffere}
{Void}
{}

\funk{void RS485\_SetAddress(uint8 addr)}
{Skifter addresse på interfacet}
{Void}
{
\funkArg{addr}{Den nye adresse der skal benyttes}
}

\funk{uint8 RS485\_GetAddress(}
{Henter adresse på interfacet}
{Adressen}
{}

\funk{uint8 RS485\_ReadRxStatus()}
{Angiver hvad status er i forhold til at der ligger en besked i bufferen}
{RS485\_MSG\_EMPTY, RS485\_MSG\_READY er de to vigtigste}
{}

\funk{void RS485\_GetRxMessage(RS485\_MSG\_STRUCT *msg)}
{Returnerer en modtaget besked, beskeden ligger stadig i bufferen og skal behandles inden den næste modtages}
{Void}
{
\funkArg{msg}{Pointer til en msg struct, som blot er pointere til forskellige steder i bufferen}
}

\funk{void RS485\_ClearRxMessage()}
{Skal køres når en besked er parset, dette gør at interfacet igen kan modtage en besked. Det gør også at værdierne i beskedn ikke længere kan forvente at være korrekte}
{Void}
{}

\funk{void  RS485\_PutTxMessage(uint8 receiver, uint8 len, uint8 cmd)}
{Sender en besked til receiver med kommandoen command, er len større end 0 skal der sendes len antal RS485\_PutTxMessageArg. Mens der sendes kan der ikke modtages beskeder. Dvs hvis der er mitchmatch mellem len og RS485\_PutTxMessageArg(uint8 arg) ender programmet i en deadlock}
{Void}
{
\funkArg{receiver}{Modtager adressen af beskeden}
\funkArg{len}{Antallet af argumenter der sendes med}
\funkArg{cmd}{Kommandoen der sendes med}
}

\funk{void RS485\_DebugHandle(const char ch)}
{Ikke implementeret endnu }
{Void}
{
\funkArg{ch}{Input char}
}

\funk{void RS485\_DebugMsg(RS485\_MSG\_STRUCT *msg)}
{Udskriver beskeden på skærmen i HEX koder, hvis debug er aktiveret bliver alle modtagne pakker udskrevet på denne måde}
{Void}
{
\funkArg{msg}{Pointer til beskeden}
}

\subsection{SensorBus}
\subsubsection{Formål}
SensorBussen er en overbygning på en I2C Multi-Master-Slave Bus. 
Der bør kun sidde en Sensor Ø på bussen, men op til 10 Fieldsensors 
(begrænset i Sensor Ø’ens hukommelse). Hovedformålet er at hive 
sensordata ud af Fieldsensorne, derfor er SensorPoll bygget op til 
kun at hive måleværdier ud som default. Hvis Fieldsensortypen er ukendt, 
kan denne også trække ud, bemærk at der bliver skiftet tilbage til Value, 
derved er der mindst muligt overhead i de fremtidige afmålinger. 
Der er startet op på en Scan funktion, samt en automatisk tildeling 
af adresser til Fieldsensor. Disse er ikke blevet færdigjort, derfor 
heller ikke dokumenteret i nedenstående. Scan benytter sig af 
non-blocking read, til at teste om en enhed svarer.

\SekvensDiagram{0.82}{AVSLibrary}{SensorPoll}

\subsubsection{Funktioner}

\funk{void SB\_Start(void)}
{Initialisere I2C komponenten, samt indlæser dummy værdier indtil reelle værdier er klar.}
{Void}
{}

\funk{void SB\_Communicate()}
{Er funktionen der kalder ParseRead og ParseWrite, denne funktion bør kaldes i mainloopet}
{Void}
{}

\funk{uint8 SB\_Read(uint8 addr, uint8 * str, uint8 len, uint8 blocking\_mode)}
{Der er to modes til denne kommando blocking og 
non-blocking. I blocking lader den hardwaren om al 
kommunikation, dog kan dette ende i deadlock. 
i non-blocking mode tester den om enheden eksisterer 
og kan kommunikeres med.}
{Success: 1 Failure: Anything else}
{
\funkArg{addr}{Adressen der skal læses fra}
\funkArg{str}{Char pointer hvor dataen fra læsningen gemmes}
\funkArg{len}{Hvor mange chars der skal læses}
\funkArg{blocking\_mode}{Sætter blocking mode}
}

\funk{uint8 SB\_Write(uint8 addr, uint8 * str, uint8 len, uint8 blocking\_mode)}
{Se SB\_Read}
{Success: 1 Failure: Anything else}
{
\funkArg{addr}{Adressen der skal skrives til}
\funkArg{str}{Char pointer hvor dataen fra skrivningen hentes}
\funkArg{len}{Hvor mange chars der skal skrives}
\funkArg{blocking\_mode}{Sætter blocking mode}
}

\funk{void SB\_ParseRead(void)}
{Nulstiller readpointeren efter læsning.}
{Void}
{}

\funk{void SB\_ParseWrite(void)}
{Fieldsensor: Skifter hvilken værdi der skal returneres, hvis Sensor Øen har bedt om dette. Sensor Ø: Bruges til at verificere at Fieldsensoren har skiftet mode}
{Void}
{}

\funk{void SB\_ChangeMode(uint8 mode)}
{Fieldsensor: Skifter mode mellem Value og Type}
{Void}
{
\funkArg{mode}{SB\_REQ\_TYPE, SB\_REQ\_VALUE}
}

\funk{void SB\_RefreshData()}
{Opdatere read bufferen med Type eller Value værdierne}
{Void}
{}

\funk{void SB\_LoadValue(uint8 high, uint8 low)}
{Bruges af Fieldsensoren til at opdatere værdien der måles}
{Void}
{
\funkArg{high}{0x00 - 0xFF Værdien har betydning efter hvilken Fieldsensor der anvendes}
\funkArg{low}{0x00 - 0xFF Værdien har betydning efter hvilken Fieldsensor der anvendes}
}

\funk{void SB\_SensorChangeMode(uint8 addr, uint8 mode)}
{Sensor Ø: Få en Fieldsensor til at skifte mode}
{Void}
{
\funkArg{addr}{Adressen på Fieldsensoren}
\funkArg{mode}{Hvilken mode der skal skiftes til}
}

\funk{void SB\_SensorPoll(uint8 addr, uint8 *values, uint8 *type)}
{Sensor Ø: Henter værdier samt type fra en Fieldsensor}
{Success: 1 - Failure: 0}
{
\funkArg{addr}{Adressen på Fieldsensoren}
\funkArg{values}{uint8 pointer hvor values gemmes i}
\funkArg{type}{uint8 pointer hvor typen gemmes i}
}

\funk{void SB\_DebugHandle(const char ch)}
{Kan bl.a. læse og skrive I2C direkte, læse i buffers, scanne efter enheder mm. Debug handler,
se mere under DebugUart}
{Void}
{
\funkArg{ch}{Input char}
}

\subsection{Ventil}
\subsubsection{Formål}
Formålet med denne komponent er at lave en ens 
tilgang til alle ventiler. Den indeholder en Pin, 
som kan sættes logisk høj (1) eller logisk lav(0). 
Pin signalet bliver routed til en udgang på psoc'en, 
som den styrekredsen til Ventil kan aflæse og derefter 
ændre ventil staten. I bund og grund en simpel wrapper 
for en Pin, men med debug indbygget.

\subsubsection{Funktioner}

\funk{void Ventil\_Start(void)}
{Funktionen bruges til at initialisere Pin komponenten, og derved sikre at ventilen er lukket}
{Void}
{}

\funk{void Ventil\_SetState(uint8 state)}
{Funktionen bruges til at sende signal til at sætte staten på Pin komponenten, staten bliver forwardet direkte til denne}
{Void}
{
\funkArg{state}{0: Lukket 1: Åben}
}

\funk{uint8 state Ventil\_GetState(void)}
{Funktionen bruges til at aflæse Pin'ens state}
{0: Lukket 1: Åben}
{}

\funk{void Ventil\_DebugHandle(const char ch)}
{Debug handler, se mere under DebugUart}
{Void}
{
\funkArg{ch}{Input char}
}

\funk{void Ventil\_DebugState(void)}
{Udskriver hvilken state ventilen er i}
{Void}
{}

\subsection{StubProjects}
\subsubsection{Formål}
Der blev udviklet en del stub projekter, enten til test af komponenter eller kommunikation.

\subsubsection{Debug Tester}
Projekt til at teste Debug komponenten løbende og at den havde den ønskede funktionalitet

\subsubsection{Fieldsensor Dummy}
Projekt der kunne agere som en Fieldsensor, der blot sendte statiske værdier tilbage, blev brugt til test af sensor ø'er for at sikre at ændringer ikke påvirkede kommunikationen mellem Fieldsensor og Sensor Øen.

\subsubsection{Fieldsensor Tester}
Projekt der blev brugt til at teste Fieldsensorer efter de blev bygget, for at teste om de gav korrekte værdier tilbage og om Fieldsensoren endte i deadlocks mm.

\subsubsection{FlowSensor Dummy}
Projekt til at teste og udvikle Flowsensor komponenten med udenfor Karet.

\subsubsection{pHProbe Dummy}
Projekt til at teste og udvikle pHProben komponenten med udenfor Karet.

\subsubsection{RS485 Dummy}
Projekt til at agere endpoint i RS485 kommunikations tests, den har få kommandoer indkodet og er til at teste selve protokollen og ikke kommandoerne. Derved kunne der testes fra andre platforme før selve Karret var færdig bygget. Blev også brugt til at udvikle RS485 komponenten.

\subsubsection{SensorBus Dummy}
Projekt til at agere endpoint i SensorBus kommunikations tests. Blev også brugt til at udvikle SensorBus komponenten.