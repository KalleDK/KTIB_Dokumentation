\subsubsection{Formål}
Denne komponent er en UART udbygget til at virke via en RS485 bus. 
Der er tilføjet en TX Pin der skal være høj under transmission, og 
ellers lav. UARTen er bygget via interrupts og buffer, 
hvilket sikre bedre mod tab af beskedr. TX benet er implementeret ved 
at disable RX og enable TX, tænde for TX Pin og vente 20 ms. Derefter 
sende komandoen samt eventuelle argumenter. Når alt er sendt disables 
TX Pin, samt enables RX og TX disables. Da der ingen collison detection 
er, er protokollen lavet til at kun masteren snakker, og slaves kun 
svarer når master forventer dette. Modtagelsen af beskeder sker via interrupts 
og det er først relevant at kalde RS485\_GetRxMessage når beskeden er helt 
modtaget. Mark Space adresseringen er indbygget i UART komponenten, eller 
er modtage statemachinen ca det samme som kan ses under \ref{fig:RS485RX_StateDiagram}. 
Afsendelsen er lidt mere kompliceret, da afsendelsen af UART beskederne sker 
i hardwaren og ikke software. Derfor er vi nød til at aktivere interrupts 
når sender bufferen er tom, og først derefter slukke for TX benet.

\StateDiagram{0.82}{AVSLibrary}{RS485TX}

\subsubsection{Funktioner}

\funk{void RS485\_Start(void)}
{Initialiserer den underlæggende UART samt 
TX Pin'en, og klargører buffere}
{Void}
{}

\funk{void RS485\_SetAddress(uint8 addr)}
{Skifter addresse på interfacet}
{Void}
{
\funkArg{addr}{Den nye adresse der skal benyttes}
}

\funk{uint8 RS485\_GetAddress(}
{Henter adresse på interfacet}
{Adressen}
{}

\funk{uint8 RS485\_ReadRxStatus()}
{Angiver hvad status er i forhold til at der ligger en besked i bufferen}
{RS485\_MSG\_EMPTY, RS485\_MSG\_READY er de to vigtigste}
{}

\funk{void RS485\_GetRxMessage(RS485\_MSG\_STRUCT *msg)}
{Returnerer en modtaget besked, beskeden ligger stadig i bufferen og skal behandles inden den næste modtages}
{Void}
{
\funkArg{msg}{Pointer til en msg struct, som blot er pointere til forskellige steder i bufferen}
}

\funk{void RS485\_ClearRxMessage()}
{Skal køres når en besked er parset, dette gør at interfacet igen kan modtage en besked. Det gør også at værdierne i beskedn ikke længere kan forvente at være korrekte}
{Void}
{}

\funk{void  RS485\_PutTxMessage(uint8 receiver, uint8 len, uint8 cmd)}
{Sender en besked til receiver med kommandoen command, er len større end 0 skal der sendes len antal RS485\_PutTxMessageArg. Mens der sendes kan der ikke modtages beskeder. Dvs hvis der er mitchmatch mellem len og RS485\_PutTxMessageArg(uint8 arg) ender programmet i en deadlock}
{Void}
{
\funkArg{receiver}{Modtager adressen af beskeden}
\funkArg{len}{Antallet af argumenter der sendes med}
\funkArg{cmd}{Kommandoen der sendes med}
}

\funk{void RS485\_DebugHandle(const char ch)}
{Ikke implementeret endnu }
{Void}
{
\funkArg{ch}{Input char}
}

\funk{void RS485\_DebugMsg(RS485\_MSG\_STRUCT *msg)}
{Udskriver beskeden på skærmen i HEX koder, hvis debug er aktiveret bliver alle modtagne pakker udskrevet på denne måde}
{Void}
{
\funkArg{msg}{Pointer til beskeden}
}