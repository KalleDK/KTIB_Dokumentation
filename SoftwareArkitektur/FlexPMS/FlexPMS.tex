\section{FlexPMS}
FlexPMS er designet til at være bindeled mellem brugergrænsefladen og kar/sensorø-styringerne. Dertil får programmet nogle opgaver der hedder sig, at det skal opsample data og gemme dette i databasen, samt at holde en log af det opsamlede data. Programmet skal også facilitere kommunikationen fra brugergrænsefladen til kar/sensorø når der for eksempel bliver bedt om at starte en vanding, eller det ønskes at styre ventiler. FlexPMS kunne også udvides således at programmet selv kan bestemme hvornår der skal vandes baseret på de målinger der er fortaget.\\\\

Herunder ses et pakkediagram, der viser de forskellige dele af systemet. Hver enkelt pakke er beskrevet i følgende afsnit, hvor der også kan findes beskrivelser af de klasser, der bor i de individuelle pakker.

\PakkeDiagram{1}{FlexPMS}{FlexPMS}

\subsection{Use cases}

\subsubsection{Start manuel vanding}

Bla bla bla

\begin{figure}[H]
	\centering
	\includegraphics[scale=.6]{SoftwareArkitektur/FlexPMS/Diagrammer/Case_StartManuelVanding.png}
	\caption{FlexPMS' håndtering af at starte manuel vanding}
	\label{photo:OpenOValveUseCase}
\end{figure}


\subsubsection{Stop manuel vanding}

Bla bla bla

\begin{figure}[H]
	\centering
	\includegraphics[scale=.6]{SoftwareArkitektur/FlexPMS/Diagrammer/Case_StopManuelVanding.png}
	\caption{FlexPMS' håndtering af at stoppe manuel vanding}
	\label{photo:OpenOValveUseCase}
\end{figure}


\subsubsection{Åben afløbsventil}

Bla bla bla

\begin{figure}[H]
	\centering
	\includegraphics[scale=.6]{SoftwareArkitektur/FlexPMS/Diagrammer/Case_OpenOValve.png}
	\caption{FlexPMS' håndtering af at åbne afløbsventilen}
	\label{photo:OpenOValveUseCase}
\end{figure}


\subsubsection{Luk afløbsventil}

Bla bla bla

\begin{figure}[H]
	\centering
	\includegraphics[scale=.6]{SoftwareArkitektur/FlexPMS/Diagrammer/Case_CloseOValve.png}
	\caption{FlexPMS' håndtering af at lukke afløbsventilen}
	\label{photo:CloseOValveUseCase}
\end{figure}


\subsubsection{Åben indløbsventil}

Bla bla bla

\begin{figure}[H]
	\centering
	\includegraphics[scale=.6]{SoftwareArkitektur/FlexPMS/Diagrammer/Case_OpenIValve.png}
	\caption{FlexPMS' håndtering af at åbne indløbsventilen}
	\label{photo:OpenOValveUseCase}
\end{figure}


\subsubsection{Luk indløbsventil}

Bla bla bla

\begin{figure}[H]
	\centering
	\includegraphics[scale=.6]{SoftwareArkitektur/FlexPMS/Diagrammer/Case_CloseIValve.png}
	\caption{FlexPMS' håndtering af at lukke indløbsventilen}
	\label{photo:CloseOValveUseCase}
\end{figure}



\subsection{Threading}
FlexPMS er dybt afhængig af trådteknologi. De tre store komponenter, Socket serveren, Bridge og Kar bus, kører parallelt i hver sin tråd. Trådene kommunikerer med hinanden gennem et event-baseret beskedsystem. Al trådhåndtering er skrevet specifikt til at køre på Linux, og FlexPMS er derfor ikke understøttet af andre operativsystemer.

\KlasseDiagram{0.5}{FlexPMS}{Thread}

\subsubsection{Thread}
\textit{Thread} er en abstrakt basis-klasse for alle klasser, som skal afvikles i sin egen tråd. Ved at nedarve fra \textit{Thread} kan en klasse nøjes med at implementere en \textKode{run()} metode, der kaldes, når tråden startes via \textKode{start()}. Tråden lever indtil \textKode{run()} returnerer, eller indtil der kaldes \textKode{cancel()} på en tråd, og tråden dør. \textit{Thread} er udelukkende skrevet til at understøtte pthread på Linux.

\StateDiagram{0.6}{FlexPMS}{Thread}

\subsubsection{Arkitekturspecifikke metoder}

\funk{void start()}{Starter eksekvering af tråden}{Ingenting}
{}

\funk{void cancel()}{Stopper tråden, hvis annullering er slået til, ellers gør funktionen ingenting. Tråden stoppes først når der stødes på et såkaldt cancellation point. Kun tråden selv kan tillade eller forbyde annullering}{Ingenting}
{}

\funk{void join()}{Blokerende kald, som ikke returnerer før tråden er færdig med eksekvering}{Ingenting}
{}

\funk{virtual void run()}{Abstrakt metode, som kaldes når tråden startes. Tråden lever så længe \textKode{run()} er under afvikling, eller indtil den annulleres}{Ingenting}
{}

\funk{static void* run\_thread(void* arg)}{C-style funktion hvori tråden startes. Denne funktion kaldes af \textKode{start()} og kalder til gengæld \textKode{run()} på Thread-objektet}{Ingenting}
{
\funkArg{arg}{En pointer til det Thread-objekt, som skal køres i en tråd}
}

\funk{void enable\_cancel()}{Tillader annullering af tråden, så tråden kan stoppes hvis \textKode{cancel()} kaldes}{Ingenting}
{}

\funk{void disable\_cancel()}{Forbyder annullering af tråden, så kald til \textKode{cancel()} ignoreres}{Ingenting}
{}

\funk{void ssleep(unsigned int sec)}{Lægger tråden til at sove i et antal sekunder (minimum)}{Ingenting}
{
\funkArg{sec}{Antal sekunder tråden minimum skal sove i}
}

\funk{void msleep(unsigned int msec)}{Lægger tråden til at sove i et antal millisekunder (minimum)}{Ingenting}
{
\funkArg{msec}{Antal millisekunder tråden minimum skal sove i}
}


\subsection{Event-baseret beskedsystem}
FlexPMS er opbygget af adskillige tråde, som alle kan snakke sammen ved at sende beskeder til hinanden. Trådene håndterer udelukkende beskeder sendt til dem udefra, og står derfor udelukkende i blokerende kald til en besked-kø (et \textit{MessageQueue}-objekt) så længe de ikke er ved at håndtere en indkommende besked. Trådene nedarver fra \textit{MessageThread} og har pointers til de tråde, som de skal kunne sende beskeder til.

\KlasseDiagram{1}{FlexPMS}{EventSystem}


\subsubsection{MessageThread}
Klassen, som er en specialisering af \textit{Thread}, stiller funktionalitet til rådighed til at indgå i det event-baserede beskedsystem. Ved at nedarve fra \textit{MessageThread} bliver en klasse til en modtager af beskeder, og kan i den forbindelse nøjes med at implementere en \textKode{dispatch()} metode, som kaldes hver gang tråden modtager en besked via dens \textKode{send()} metode. \textKode{dispatch()} modtager to argumenter; et event-ID samt en pointer til et \textit{Message}-objekt, der evt. kan være \textKode{NULL}. Dispatch bør overholde reglen om, at kalde en funktion til at håndtere beskeden alt efter hvilket event-ID den modtager, hvilket typisk implementeres vha. en switch-case.\\\\

\textKode{send()} tager ligeledes to argumenter; et event-ID samt en pointer til et \textit{Message}-objekt. Det er afsenderen, som skal allokere \textit{Message}-objektet, men \textit{MessageThread} sørger selv for, at de-allokere det efter \textKode{dispatch()} er kaldt hos modtageren, og beskeden er håndteret.\\\\

Nedenfor ses et eksempel på kommunikation mellem to tråde, \textit{MessageThread1} og \textit{MessageThread2}, initieret af \textit{MessageThread1}, hvor \textit{MessageThread2} svarer på beskeden. 

\SekvensDiagram{1}{FlexPMS}{MessageThread}

\textit{MessageThread} laver udelukkende blokerende kald til dens besked-kø (et \textit{MessageQueue}-objekt), og dermed undgår vi at stå og bruge CPU-tid i løkker, som ikke udfører noget. Det betyder, at alle klasse som nedarver fra \textit{MessageThread} udelukkende håndterer events sendt til dem udefra. På den måde lægges tråde til at sove så længe der ikke er noget at lave, og programmet vil bruge minimalt CPU-tid.


\StateDiagram{0.5}{FlexPMS}{MessageThread}

\subsubsection{Arkitekturspecifikke metoder}

\funk{virtual void init()}{Abstrakt metode, som kaldes inden der begyndes at hente beskeder fra beskedkøen. Kan bruges til at lave initiering af objektet. Metoden kan overskrives af klasser, som nedarver fra \textit{MessageThread}}{Ingenting}
{}

\funk{void send(unsigned long id, Message* msg = NULL)}{Sender en besked til tråden, ved at lægge en besked i dens beskedkø}{Ingenting}
{
\funkArg{id}{Et ID, som beskriver det event der sendes}
\funkArg{msg}{En pointer til et \textit{Message}-objekt, som kan holde på yderligere data}
}

\funk{virtual void dispatch(unsigned long event\_id, Message* msg))}{Abstrakt metode, som kaldes hver gang tråden modtager en besked. Metoden skal overskrives af klasser, som nedarver fra \textit{MessageThread} til at håndtere indkommende beskeder}{Ingenting}
{
\funkArg{event\_id}{Et ID, som beskriver det event der sendes}
\funkArg{msg}{En pointer til et \textit{Message}-objekt, som kan holde på yderligere data. Kan være \textKode{NULL}}
}


\subsubsection{MessageQueue}

Klassen er en FIFO kø, som er trådsikret, dvs. sikret mod de problemer der kan opstå i forbindelse med at tilgå den parallelt fra forskellige tråde. \textit{MessageQueue} er implementeret via en queue (fra STL) og benytter sig at pthread’s \textit{mutex} og \textit{conditional variable} til at synkronisere mellem tråde. \textit{MessageQueue} er udelukkende brugt internt i \textit{MessageThread}.

\begin{figure}[H]
	\centering
	\includegraphics[scale=1]{SoftwareArkitektur/FlexPMS/Diagrammer/MessageQueue_FIFO.png}
	\caption{MessageQueue FIFO}
	\label{photo:MessageQueueFIFO}
\end{figure}


\subsubsection{Arkitekturspecifikke metoder}

\funk{void send(unsigned long id, Message* msg = NULL)}{Putter en besked i køen}{Ingenting}
{
\funkArg{id}{Event-ID som skal puttes i køen}
\funkArg{msg}{Pointer til \textit{Message}-objekt, som skal puttes i køen}
}

\funk{Message* recieve(unsigned long\& id)}{Henter den næste besked fra køen. Hvis køen er tom, så blokerer funktionen indtil der bliver puttet noget i køen via \textKode{send()}}{En pointer til et \textKode{Message}-objekt. Kan være \textKode{NULL}}
{
\funkArg{id}{Funktionen skriver event-ID til denne variabel}
}

\subsubsection{Item}

En struct, som udelukkende benyttes internt i \textit{MessageQueue}. Den holder på event ID'er og \textit{Message}-objekter, og er den type, der placeres i køen.


\subsubsection{Message}

\textit{Message} gør det muligt at medsende informationer (ud over et event-ID), når tråde kommunikerer med hinanden. Klassen \textit{Message} indeholder en \textKode{sender} attribut, som er en pointer til det \textit{MessageThread}-objekt, der sendte beskeden. \textKode{sender} gør det derfor muligt for \textit{MessageThread}'s at svare på beskeder uden at kende til afsenderen.\\\\

Alt efter hvilke data man vil sende med en besked kan man nedarve fra \textit{Message} og tilføje flere attributter.  Der er lavet specialiseringer af \textit{Message} hvor det har været nødvendigt at sende information mellem tråde.\\\\

\textbf{KarBusMessage}\\
Klassen er en specialisering af \textit{Message} og indeholder – ud over \textKode{sender} – en \textKode{kar} attribut, som er en pointer til det \textit{Kar}-objekt, der skal sendes data til. Grunden til, at \textKode{kar} kun er implementeret på \textit{KarBusMessage} er, at det altid er relevant at have information med omkring hvilket kar der skal sendes instrukser til når der kommunikeres mellem \textit{Bridge} og \textit{KarBus}, hvorimod dette ikke er nødvendigt når der kommunikeres mellem \textit{Bridge} og \textit{SocketClient}.\\
Der er adskillige specialiseringer af \textit{KarBusMessage}, som benyttes alt afhængigt af beskeden (event-ID).\\\\

\textbf{SessionMessage}\\
Klassen er en specialisering af \textit{Message} og bruges i forbindelse med, at \textit{SocketClient} skal registrere sig selv hos \textit{Bridge}. \textit{Bridge} sender en \textit{SessionMessage} til \textit{SocketClient} når klienten er registreret. Klassen indeholder en \textKode{session\_id} attribut.\\\\

\textbf{GuiMessage}\\
Klassen er en specialisering af \textit{Message} og indeholder – ud over sender – flere attributter, som bruges i forbindelse med kommunikation mellem \textit{SocketClient} og \textit{Bridge}. Klienten (webserveren) har mulighed for at sende kommandoer, som er specifikke for enten Kar eller SensorØ, og det er derfor relevant at tilføje disse attributter på \textit{GuiMessage}. Klassen indeholder desuden også en \textKode{session\_id} attribut, så Bridge ved hvilken klient den skal sende til i tilfælde, hvor svar til klienten er nødvendigt.


\subsection{Logging}
Vi benytter logging til at gemme alle handlinger, som har med sensorer og aktuatorer at gøre. Loggen skrives til en fil med tidsstempel. Det er udelukkende Bridge, som skriver til loggen.\\\\

Følgende handlinger bliver logget:

\begin{itemize}
\item Der modtages data fra sensorer på et Kar
\item Der modtages data fra sensorer på en SensorØ
\item Der modtages information om, hvorvidt en ventil på et Kar er blevet åbnet eller lukket
\item Der modtages information om, hvorvidt en ventil på en SensorØ er blevet åbnet eller lukket
\item Der modtages information om en pumpes status (slukket eller tændt/pumpe-hastighed)
\item Når GUI anmoder om at starte manuel vanding logges der, for hver SensorØ, at dens ventil blev anmodet om at åbne. Der logges, at Karrets pumpe blev anmodet om at startet
\item Når GUI anmoder om at stoppe manuel vanding logges der, for hver SensorØ, at dens ventil blev anmodet om at lukke. Der logges, at Karrets pumpe blev anmodet om at stoppet
\item Når GUI anmoder om at åbne eller lukke indløbs- eller afløbsventil på et Kar
\end{itemize}

Alt andet debugging skrives til FlexPMS' stdout. Det er derfor muligt at omdirigere stdout til en fil for at logge samtlige debugging udskrifter.


\KlasseDiagram{0.5}{FlexPMS}{Logging}

\subsubsection{Log}
Klassen benyttes til at skrive til loggen. Den er implementeret vha. Singleton, dvs. at kun én instans af klassen kan eksistere på tværs af hele FlexPMS. Instansen bliver oprettet gennem kald fra main.cpp, hvorefter den lever som en statisk member på klassen Log så længe FlexPMS kører.\\\\

Hver gang der skrives til loggen tilføjes tidsstempel samt linjeskift efter teksten.

\subsubsection{Arkitekturspecifikke metoder}

\funk{static Log* getInstance()}{Funktionen returnerer en instans af Log. Hvis en instans endnu ikke er blevet oprettet, så oprettes den først}{Instans af klassen Log}
{}

\funk{void write(std::string line)}{Funktionen skriver en linje til logfilen}{Ingenting}
{
\funkArg{line}{Tekst, som skal skrives til logfilen}
}



\subsection{Database og domænemodeller}
Systemet benytter sig af en MySQL database. For at forbinde til databasen gennem FlexPMS benyttes det officielle bibliotek fra MySQL til at forbinde gennem C++, \textit{mysqlcppconn}\footnote{\citet{oracle:mysqlcppconn}}. FlexPMS kører i øvrigt ikke transaktionsbaseret.\\\\

Tilgang til databasen er pakket ind i forskellige domæneklasser, som håndterer de lavpraktiske SQL forespørgsler. Domæneklasserne stiller en højniveau-grænseflade til rådighed for resten af programmet.\\\\

Forbindelsen til databasen oprettes i main.cpp når FlexPMS starter, og holdes åben så længe programmet lever. Der gives en pointer til DB-forbindelsen med til \textit{Bridge}, som er den eneste klasse, der arbejder med domæneklasserne.


\KlasseDiagram{1}{FlexPMS}{Database}

\subsubsection{Kar}
Klassen repræsenterer ét Kar, og stiller en højniveau-grænseflade til rådighed, til at opdatere værdier for et kar i databasen.

\subsubsection{Arkitekturspecifikke metoder}

\funk{void set\_mwstatus(bool s)}{Funktionen sætter status for manuel vanding}{Ingenting}
{
\funkArg{s}{Status for manuel vanding. \textKode{True} er tændt, \textKode{False} er slukket}
}

\funk{void set\_ivalvestatus(bool s)}{Funktionen sætter status for, hvorvidt indløbsventilen er åben eller lukket}{Ingenting}
{
\funkArg{s}{Status for indløbsventilen. \textKode{True} er åben, \textKode{False} er lukket}
}

\funk{void set\_ovalvestatus(bool s)}{Funktionen sætter status for, hvorvidt afløbsventilen er åben eller lukket}{Ingenting}
{
\funkArg{s}{Status for afløbsventilen. \textKode{True} er åben, \textKode{False} er lukket}
}

\funk{void add\_sensor\_data(int type, double value)}{Funktionen registrerer data fra en sensor koblet til karret}{Ingenting}
{
\funkArg{type}{Sensor type ID. Se Tabel \ref{table:karSensorData_kol} for mulige værdier}
\funkArg{value}{Den målte værdi}
}


\subsubsection{SensorOe}
Klassen repræsenterer én sensor ø, stiller en højniveau-grænseflade til rådighed, til at opdatere værdier for en sensor ø i databasen.

\subsubsection{Arkitekturspecifikke metoder}

\funk{void set\_valvestatus(bool s)}{Funktionen sætter status for, hvorvidt ventilen er åben eller lukket}{Ingenting}
{
\funkArg{s}{Status for ventilen. \textKode{True} er åben, \textKode{False} er lukket}
}

\funk{void add\_sensor\_data(int type, double value)}{Funktionen registrerer data fra en sensor koblet til sensor ø’en}{Ingenting}
{
\funkArg{type}{Sensor type ID. Se Tabel \ref{table:oeSensorData_kol} for mulige værdier}
\funkArg{value}{Den målte værdi}
}



\subsubsection{DBContainer}
\textit{DBContainer} er en abstrakt basis-klasse for alle klasser, som skal holde på lister af objekter. \textit{DBContainer} implementerer funktionalitet til at tilgå en række af resultater fra et databaseudtræk. Den nedarvede klasse skal selv implementere databaseudtrækket via \textKode{reload()} metoden, men får derudover stillet alle andre nødvendige metoder til rådighed til at tilgå resultaterne.\\\\

\textit{DBContainer} er implementeret med et C++ map (fra STL), hvor nøglen er et unikt ID og værdien er et objekt, som repræsenterer én række i databasen. Da \textit{DBContainer} er en template-klasse kan objekterne være af hvilken som helst type. Nøglerne skal dog være af typen \textKode{unsigned int}.\\\\

Nedarvede klasser skal implementere \textKode{reload()} metoden, som kaldes umiddelbart efter at objektet er blevet instantieret. \textKode{reload()} metodens formål er, at hente data fra databasen, instantiere objekter og tilføje dem til mappet.

\subsubsection{Arkitekturspecifikke metoder}

\funk{T* get(unsigned int id)}{Returnerer en pointer til objektet med nøglen \textKode{id}. Hvis nøglen ikke findes returneres \textKode{NULL}}{En pointer til objektet med nøglen \textKode{id}. Hvis nøglen ikke findes returneres \textKode{NULL}}
{
\funkArg{id}{Primærnøglen på det objekt, som skal søges efter}
}

\funk{bool contains(unsigned int id)}{Tjekker hvorvidt et objekt eksisterer i mappet}{\textKode{True} hvis objektet eksisterer, ellers \textKode{False}}
{
\funkArg{id}{Primærnøglen på det objekt, som skal søges efter}
}

\funk{const unsigned int size()}{Tæller antallet af objekter i mappet}{Antallet af objekter i mappet}
{}

\funk{void iter()}{Forbereder objektet til at blive itereret over fra begyndelsen}{Ingenting}
{}

\funk{T* next()}{Returnerer det næste objekt i en iteration. Kald \textKode{iter()} inden iterationen begyndes for at være sikker på, at der startes fra begyndelsen}{Det næste objekt af typen \textKode{T}}
{}

\funk{virtual void reload() = 0}{Abstrakt metode, som skal implementeres af nedarvede klasser. Dens formål er, at hente data fra databasen og putte det ind i mappet. Funktionen kaldes umiddelbart efter at \textit{DBContainer} bliver instantieret}{Ingenting}
{}


\subsubsection{KarContainer}
Klasse, som er en specialisering af \textit{DBContainer}, giver mulighed for at arbejde med en liste af \textit{Kar}-objekter.


\subsubsection{SensoeOeContainer}
Klasse, som er en specialisering af \textit{DBContainer}, giver mulighed for at arbejde med en liste af \textit{SensorOe}-objekter.

\subsection{Socket server}
Denne del af FlexPMS giver GUI'en adgang til at kommunikere direkte med FlexPMS via TCP/IP, og bruges til at informere FlexPMS om handlinger, som skal startes eller stoppes, og er den eneste direkte vej for GUI at kommunikere med FlexPMS. Der kommunikeres gennem en GUI Protokol (se Tabel \ref{tab:GUIProtokol}).



\KlasseDiagram{1}{FlexPMS}{SocketServer}

\subsubsection{SocketServer}
Klassen, som er en specialisering af \textit{Thread}, har det ene formål, at lytte efter indkommende forbindelser fra GUI’en over TCP/IP. Når \textit{SocketServer} modtager en ny forbindelse startes en ny tråd, \textit{SocketClient}, hvori al kommunikation mellem GUI og FlexPMS foregår. Når \textit{SocketServer} har oprettet og startet en ny \textit{SocketClient} mister den al forbindelse til den, og kender derfor ikke til åbne forbindelser til klienter.\\\\

Serveren lytter på adresse 127.0.0.1 (localhost) port 5555.

\StateDiagram{0.5}{FlexPMS}{SocketServer}


\subsubsection{SocketClient}

\textit{SocketClient}, som er en specialisering af \textit{MessageThread}, håndterer al kommunikation mellem klienten (GUI) og \textit{Bridge}. Den bliver oprettet af \textit{SocketServer} når der kommer en ny indkommende forbindelse. Den består desuden af en privat \textit{SocketReader} klasse, hvis eneste formål er, at læse data fra en socket. \textit{SocketReader} er implementeret, så der kan laves blokerende læse-kald fra socket, og på den måde undgår vi, at \textit{SocketClient} står i løkker hvor den laver to ikke-blokerende kald (henholdsvis at læse fra socket, samt at læse fra sin egen besked-kø).\\\\

\textit{SocketClient} modtager beskeder fra enten \textit{SocketReader} eller \textit{Bridge}. \textit{SocketReader} sender beskeder når der enten er modtaget nyt data fra klienten, eller når klienten lukker forbindelsen. \textit{Bridge} sender beskeder når der enten er data at sende til klienten, eller når forbindelsen til klienten skal lukkes.\\\\

Når \textit{SocketClient} oprettes får den givet en file-descriptor til den socket, som den skal læse fra og skrive til. Lige efter at klassen er blevet oprettet, registrerer den sig hos \textit{Bridge}, der svarer tilbage med et unikt sessions ID, som skal gives med hver gang \textit{SocketClient} sender beskeder til \textit{Bridge}. \textit{Bridge} bruger dette ID til at identificere klienter i tilfælde, hvor der skal sendes et svar tilbage til klienten. \textit{Bridge} har derfor en intern mapning af, hvilke sessions ID'er der hører til hvilke \textit{SocketClient}-objekter. Det er med andre ord \textit{Bridge}, og ikke \textit{SocketServer}, som holder styr over åbne forbindelser til klienter.\\\\

Når \textit{SocketClient} er blevet registreret hos \textit{Bridge} starter den \textit{SocketReader}, som begynder at læse data fra socket. Herefter kan en udveksling af data mellem GUI og FlexPMS begynde.\\\\

\textit{SocketClient} dør når én af tre handlinger finder sted:

\begin{enumerate}
\item \textit{SocketReader} fik en fejl, da den forsøgte at læse fra socket
\item \textit{SocketClient} fik en fejl, da den forsøgte at skrive til socket
\item \textit{Bridge} giver besked om, at forbindelsen til klienten skal lukkes 
\end{enumerate}

I de to første tilfælde skal \textit{SocketClient} give \textit{Bridge} besked om, at forbindelse til klienten er død, og \textit{SocketClient} skal stoppes. I det sidste tilfælde skal \textit{SocketClient} reagere på beskeden og stoppe sig selv. Herefter ved \textit{Bridge}, at den skal fjerne alle spor af \textit{SocketClient} klassen (herunder session og brugt hukommelse).

\StateDiagram{1}{FlexPMS}{SocketClient}

\textbf{Events}\\

\textit{SocketClient} kan modtage følgende kommandoer fra \textit{Bridge} og \textit{SocketReader}:

\begin{itemize}
\item \textKode{E\_START\_SESSION} håndteres af \textKode{handle\_start\_session()}
\item \textKode{E\_STOP\_SESSION} håndteres af \textKode{handle\_stop\_session()}
\item \textKode{E\_KILL} håndteres af \textKode{handle\_kill()}
\item \textKode{E\_SEND\_DATA} håndteres af \textKode{handle\_send\_data()}
\item \textKode{E\_RECV\_DATA} håndteres af \textKode{handle\_recieve\_data()}
\end{itemize}

\textbf{Kommandoer}\\

\textit{SocketClient} kan modtage følgende kommandoer fra GUI gennem GUI Protokol:

\begin{itemize}
\item \textKode{MWSTART} håndteres af \textKode{handle\_start\_watering()}
\item \textKode{MWSTOP} håndteres af \textKode{handle\_stop\_watering()}
\item \textKode{IVALVEOPEN} håndteres af \textKode{handle\_ivalve\_open()}
\item \textKode{IVALVECLOSE} håndteres af \textKode{handle\_ivalve\_close()}
\item \textKode{OVALVEOPEN} håndteres af \textKode{handle\_ovalve\_open()}
\item \textKode{OVALVECLOSE} håndteres af \textKode{handle\_ovalve\_close()}
\end{itemize}

Se Tabel \ref{tab:GUIProtokol} for en detaljeret beskrivelse af kommandoerne i GUI Protokol.\\\\

Nedenfor ses et sekvensdiagram, som illustrerer \textit{SocketClient}'s livscyklus.

\SekvensDiagram{1}{FlexPMS}{SocketClient}

\subsection{Controller}
\textit{Controller}, der består af \textit{Bridge}, som er en specialisering af \textit{MessageThread}, er den centrale controller i FlexPMS. Den kommunikerer med både GUI og KarBus, og håndterer al den logik der ligger ind imellem. Det er også Bridge, der håndterer timing-baserede events som f.eks. at poll'e data fra kar. Det er også Bridge, der tilgår databasen gennem domænemodeller.

\KlasseDiagram{0.85}{FlexPMS}{Controller}

\subsubsection{Bridge}


\textbf{Sessions}
Når Bridge modtager en forespørgsel fra en \textit{SocketClient} om at blive registreret, så tildeler Bridge den forespørgende \textit{SocketClient} det næste ledige sessions ID. Herefter kan \textit{SocketClient} begynde at sende beskeder til Bridge, som videreformidler beskederne til KarBus. Beskeden til KarBus skulle indeholde et sessions ID på den forespørgende \textit{SocketClient}, så i tilfælde af, at \textit{SocketClient} skulle have svar, kunne KarBus sende ID’et med tilbage til Bridge, som kunne identificere hvilken \textit{SocketClient} den skulle sende svaret til. Vi fik dog aldrig brug for at sende svar tilbage til \textit{SocketClient}, så sessions ID'er er ikke blevet implementeret i kommunikationen mellem Bridge og KarBus.\\\\

Se Figur \ref{fig:SocketClient_SekvensDiagram} for en \textit{SocketClient}'s livscyklus, som inkluderer registrering med sessioner.\\\\

\textbf{KarPinger}
\textit{KarPinger}, som er en specialisering af \textit{Thread}, har det ene formål at sende et PING-event til \textit{Bridge} hvert 10. sekund.


\subsection{Kar Kommunikation}
Kar kommunikations pakken er ansvarlig for at sende beskeder til kar styringen den står altså for grænsefalden ud til sensor og aktautore igennem kommunikation til kar styringen. Den består af 3 klasser der er illustreret  i diagrammet herunder:

\KlasseDiagram{0.82}{FlexPMS}{FlexPMS}

\subsubsection{KarBus klassen}
Den klasse interfacer med resten af flexpms igennem den event baserede kommunikation der driver systemet klassen har således sin egen message kø og tilsvarende event handlers. KarBus har også en RS485 instans og en Protokol instans i sig, Protokol klassen er implementeringen af Karbus protokollen som KarBus klassen bruger til at sende beskeder til karret, RS485 klassen er ikke direkte brugt af KarBus.

\SekvensDiagram{0.82}{FlexPMS}{Template}

\subsubsection{funktions beskrivelser}
Duis egestas congue odio, et convallis arcu eleifend vel. Vivamus sed risus congue, tincidunt ipsum in, volutpat eros. Vestibulum eros quam, vulputate eu varius id, pellentesque at risus. Duis in iaculis dolor, consequat sollicitudin libero. Aliquam pulvinar gravida nunc vitae pulvinar. Nam ligula nisl, dapibus at facilisis vel, porttitor id metus. Curabitur malesuada enim mauris, ut dignissim sapien bibendum sit amet.

\subsubsection{Protokol klassen}
Donec venenatis luctus massa, sed interdum eros interdum quis. Morbi eget massa ut ante venenatis efficitur. Praesent tempus euismod tempus. Integer ac nisl eros. Praesent faucibus nisl non faucibus tincidunt. Pellentesque habitant morbi tristique senectus et netus et malesuada fames ac turpis egestas. Aenean tristique augue libero, interdum fringilla nibh consequat vel. Nullam id dapibus velit. Fusce sit amet dolor mollis, iaculis tellus ut, cursus nisi. \textKode{Curabitur} eleifend congue lacus sit amet ullamcorper. Pellentesque mollis tincidunt lectus sit amet cursus. Aenean euismod velit tellus, non lobortis quam dapibus a.

\SekvensDiagram{0.82}{FlexPMS}{Template}

\subsubsection{funktions beskrivelser}
Duis egestas congue odio, et convallis arcu eleifend vel. Vivamus sed risus congue, tincidunt ipsum in, volutpat eros. Vestibulum eros quam, vulputate eu varius id, pellentesque at risus. Duis in iaculis dolor, consequat sollicitudin libero. Aliquam pulvinar gravida nunc vitae pulvinar. Nam ligula nisl, dapibus at facilisis vel, porttitor id metus. Curabitur malesuada enim mauris, ut dignissim sapien bibendum sit amet.

\subsubsection{RS485 klassen}
Donec venenatis luctus massa, sed interdum eros interdum quis. Morbi eget massa ut ante venenatis efficitur. Praesent tempus euismod tempus. Integer ac nisl eros. Praesent faucibus nisl non faucibus tincidunt. Pellentesque habitant morbi tristique senectus et netus et malesuada fames ac turpis egestas. Aenean tristique augue libero, interdum fringilla nibh consequat vel. Nullam id dapibus velit. Fusce sit amet dolor mollis, iaculis tellus ut, cursus nisi. \textKode{Curabitur} eleifend congue lacus sit amet ullamcorper. Pellentesque mollis tincidunt lectus sit amet cursus. Aenean euismod velit tellus, non lobortis quam dapibus a.

\SekvensDiagram{0.82}{FlexPMS}{Template}

\subsubsection{funktions beskrivelser}
Duis egestas congue odio, et convallis arcu eleifend vel. Vivamus sed risus congue, tincidunt ipsum in, volutpat eros. Vestibulum eros quam, vulputate eu varius id, pellentesque at risus. Duis in iaculis dolor, consequat sollicitudin libero. Aliquam pulvinar gravida nunc vitae pulvinar. Nam ligula nisl, dapibus at facilisis vel, porttitor id metus. Curabitur malesuada enim mauris, ut dignissim sapien bibendum sit amet.
