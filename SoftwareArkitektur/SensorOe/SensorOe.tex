\section{Sensor Ø}

Sensor Øen består af følgende komponenter, 
hvor DebugUart er optional, og bruges under 
test og implementering. De enkelte pakker er 
videre beskrevet under AVSLibrary sektionen.

\PakkeDiagram{0.82}{SensorOe}{SensorOe}

Systemet er i kort vist i følgende sekvens diagram

\SekvensDiagram{0.82}{SensorOe}{SensorOe}

Selve parse funktionen af de enkelte beskeder er ikke 
vist da de er unikke efter hver kommando, men er simple 
i deres udførelse, f.eks. skifte state på den tilkoblede 
ventil. For at sikre mest mulig oppetid i forhold til 
Øbus kommunikationen er pollning af Fieldsensorer, med 
vilje lavet til at kun en bliver pollet per loop. Derved 
har Øbussen mulighed for kommunikation mellem hver pollning 
i stedet for at skulle vente til alle sensorer er pollet. 
Der er desuden implementeret en scannings funktion der 
automatisk tilføjer alle Fieldsensorer der er tilkoblet 
SensorBussen før Øen booter. Derved kan man tilføje de 
sensorer man skal bruge og genstarte SensorØen, de er 
derefter alle blevet tilføjet. Dette er noget der er 
blevet arbejdet på at skulle ske automatisk løbende i 
stedet for under boot.

\subsection{Funktioner}

\funk{void SetVentilState(uint8 state)}
{Bruges til at skifte tilstand på ventilen alt efter om den skal være åben eller lukket}
{intet}
{
\funkArg{state}{Den tilstand ventil skal sættes til åben(1) eller lukket(1)}
}

\funk{uint8 GetOebusAddr()}
{Bruges til at hente den adresse som øen har på ø bussen}
{Ø'ens bus adresse}
{
}

\funk{void SetOebusAddr(uint8 addr)}
{Bruges til at ændre øens adresse}
{intet}
{
\funkArg{addr}{Den adresse øen skal have}
}

\funk{void addFieldsensor(uint8 addr)}
{Denne funktion allokere plads til en ny fieldsensor og initiere den med default værdier samt den adresse funktionen er kaldt med}
{intet}
{
\funkArg{addr}{Den adresse fieldsensoren skal have}
}

\funk{void pollFieldsensor(Fieldsensor* fs)}
{Her startes udlæsning af data fra fieldsensoren, hvis det lykkes sættes sensoren til at være online ellers offline}
{intet}
{
\funkArg{fs}{Er en pointer til den sensor der skal polles}
}

\funk{void sendFieldsensors(uint8 receiver)}
{Tæller op hvor mange sensorer der er koblet til øen og sender en besked med rette retur type og længde i henhold til antal sensorer derefter deres argumenter}
{intet}
{
\funkArg{receiver}{Adressen på den der skal modtage fieldsensor data}
}

\funk{Fieldsensor* findFieldsensor(uint8 sensor\_addr)}
{Leder listen af fieldsensorer igennem efter en sensor med adressen i argumentet}
{En pointer til Fieldsensor hvis alt gik godt ellers NULL}
{
\funkArg{sensor\_addr}{Adressen på den sensor der skal findes}
}

\funk{void parseOEBUS()}
{dispatcher for OeBus}
{intet}
{
}

\funk{void scanFieldsensors(uint8 import)}
{funktionen skanner alle valide adresse og forsøger at læse fra dem hvis det lykkes importeres sensoren i fieldsensor listen såfremt import parameteren er sat}
{intet}
{
\funkArg{import}{Fortæller om der skal gemmes de sensorer der er blevet skannet}
}

\funk{void debugListFieldsensors()}
{Udskriver adresse og status på alle tilkoblede fieldsensorer på debug uart}
{intet}
{
}

\funk{void debugConfig()}
{udskriver alt data omhandlende sensorøs configuration på debug uart}
{intet}
{
}

\funk{void DebugHandle(const char ch)}
{Bruges til at skifte DebugHandle og udskrive en liste af Komponenter der er tilkoblet debug}
{intet}
{
\funkArg{ch}{tegn fra uart som der switches på}
}
>>>>>>> Beskrevet Oe funktioner
