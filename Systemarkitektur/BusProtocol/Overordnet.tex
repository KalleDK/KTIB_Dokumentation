%!TEX root = ../../main.tex

\subsection{Overordnet}
Til kommunikation i AVS har vi valgt at implementere en protokol der bruges til kommunikation mellem hardware enheder.
Det drejer sig om to busser der følger samme format, samt er der gjort brug af I2C i forhold til følere. De to busser Kar og Ø bus basere sig begge på samme data format da deres fysiske lag er ens. Busserne kører master/slave kommunikation det gør at det kun er master der kan initiere kommunikation.\\\\

Til at kommunikere mellem GUI og CentralControl har vi designet endnu en protokol, som er uafhængig af ovenstående.

\subsection{Dataformat - Kar og Ø bus} 
Data bliver sendt på følgende format:

\begin{enumerate}
\item Adressen på modtager af besked
\item Adressen på afsender af besked
\item Længden af de kommende argumenter (kan være 0)
\item Kommando der ønskes behandlet
\item Argumenter i henhold til længden
\end{enumerate}

Systemet er inddelt i forespørgsler og svar, grundet master/slave formatet. Derfor kan en hel besked betragtes som en forespørgsel fra master og et svar fra slave.

Forespørgsler og svar har samme format som beskrevet ovenfor.

En kommunikation mellem master og slave kunne se således ud:
Master vil sende på dette format:
\begin{table}[H]
\setlength{\parindent}{12pt}
\begin{tabular}{|l|l|c|c|c|}\hline
\multicolumn{5}{|l|}{Master\cellcolor[gray]{0.9}}\\\hline
RxAddr & TxAddr & len & cmd & args \\\hline
1 Byte & 1 Byte & 1 Byte & 1 Byte & len * 1 Bytes \\\hline 
\end{tabular}
\end{table}


Slave vil svare på dette format:
\begin{table}[H]
\setlength{\parindent}{12pt}
\begin{tabular}{|l|l|c|c|c|}\hline
\multicolumn{5}{|l|}{Slave\cellcolor[gray]{0.7}}\\\hline
RxAddr & TxAddr & len & cmd & args  \\\hline
1 Byte & 1 Byte & 1 Byte & 1 Byte & len * 1 bytes \\\hline 
\end{tabular}
\end{table}

Alt efter hvilken kommando der sendes vil det variere om der er argumenter med det vil sige at len godt kan være 0. Som det også kan ses er master/slave forespørgsel og svar ens i opbygningen.

\subsection{Kar bus kommandoer}
Alle kommandoer er delt op på et "request" / "response" format dette er pga. af master/slave kommunikationen, det gør så at master altid kan forvente at få et svare og hvis ikke der bliver svaret er der sket en fejl.

Der er implementere følgende kommandoer på KarBus:

\begin{table}[H]
\setlength{\parindent}{12pt}
\begin{tabular}{|l|l|}\hline
\multicolumn{2}{|c|}{KarBus\cellcolor[gray]{0.7}}\\\hline
Kommando & Beskrivelse \\\hline
REQ\_KAR\_SENSOR\_DATA 		& Forespørgsel på at få data fra sensor der er koblet til kar \\\hline 
RES\_KAR\_SENSOR\_DATA 		& Svar indeholdende sensor data								 \\\hline 
REQ\_KAR\_VENTIL	   		& Forespørgsel på at skifte tilstand på ventil der er koblet til kar \\\hline 
RES\_KAR\_VENTIL       		& Svar der indeholder den tilstand ventilen er sat til \\\hline 
REQ\_KAR\_PUMPE		   		& Forespørgsel på at skifte hastighed på pumpen \\\hline 
RES\_KAR\_PUMPE 	   		& Svar indeholdende den hastighed pumpen der sat til \\\hline 
REQ\_KAR\_OPRET 	   		& Forespørgsel om at oprette en ny sensor ø \\\hline
RES\_KAR\_OPRET 	   		& Svar med adressen på den nye ø \\\hline 
REQ\_KAR\_OE\_LIST	   		& Forespørgsel om at oprette en ny sensor ø \\\hline
RES\_KAR\_OE\_LIST	   		& Svar med adressen på den nye ø \\\hline 
REQ\_KAR\_OE\_SENSOR\_DATA	& Forespørgsel på at få sensors øens sensor data \\\hline
RES\_KAR\_OE\_SENSOR\_DATA 	& Svar med data for sensor ø og om sensor er koblet til eller ej \\\hline 
REQ\_KAR\_OE\_VENTIL 	    & Forespørgsel om at åbne/lukke sensor øens ventil \\\hline
RES\_KAR\_OE\_VENTIL 	    & Svar med den tilstand ventilen er efterlad i \\\hline
REQ\_KAR\_OE\_SENSOR\_TYPE 	& Forespørgsel om hvilken type føler der er tilsluttet øen \\\hline
RES\_KAR\_OE\_SENSOR\_TYPE 	& Svar med sensor type \\\hline  
\end{tabular}
\end{table}

\subsubsection{REQ\_KAR\_SENSOR\_DATA}
Denne kommando er en forespørgsel på data som Karrets sensorer har indsamlet.

\begin{table}[H]
\setlength{\parindent}{12pt}
\begin{tabular}{|l|lcc|}
\hline
\multicolumn{4}{|c|}{Kommado med argumenter}\\\hline
Kommando & 0x1 & & \\
Argumenter & ingen & & \\\hline
\end{tabular}
\end{table}



\subsubsection{RES\_KAR\_SENSOR\_DATA}
Denne kommando er et svar der indeholder den data karrets sensorer har indsamlet.

\begin{table}[H]
\setlength{\parindent}{12pt}
\begin{tabular}{|l|lcc|}
\hline
\multicolumn{4}{|c|}{Kommado med argumenter}\\\hline
Kommando & 0x2 & & \\
Argumenter & ID & Value1 & Value2 \\\hline
\end{tabular}
\end{table}

Argumenterne kan forekomme flere gange og vil være reflekteret i længden. Det vil sige at en længde på 3 i dette tilfælde betyder at der kun er indeholdt en sensor i svaret, hvis længden var 6 ville der være to sensorer osv.

\begin{table}[H]
\setlength{\parindent}{12pt}
\begin{tabular}{|l|l|}
\hline
\multicolumn{2}{|c|}{Argument uddybning}\\\hline
ID & Dette er følerens identifikation alt efter om det er pH, flow eller andet der bliver målt. \\
Value1 & Er heltals værdien af den målte størrelse. \\
Value2 & Hvis MSB'en i denne er sat betyder det at der skal lægges en halv til value1.\\\hline
\end{tabular}
\end{table}


\subsubsection{REQ\_KAR\_VENTIL}
Denne kommando er en forespørgsel på at styre karrets ventiler.

\begin{table}[H]
\setlength{\parindent}{12pt}
\begin{tabular}{|l|lcc|}
\hline
\multicolumn{4}{|c|}{Kommado med argumenter}\\\hline
Kommando & 0x3 & & \\
Argumenter & ID & STATE & \\\hline
\end{tabular}
\end{table}

Her er ID den ventil der ønske manipuleret og STATE er hvilken tilstand der ønskes at ventilen skal stå i.\\\\
STATE kan være følgende:
\begin{table}[H]
\setlength{\parindent}{12pt}
\begin{tabular}{|l|l|}
\hline
\multicolumn{2}{|c|}{Argumenter uddybet}\\\hline
Lukket & 0x0 \\
Åben & 0x1 \\\hline
\end{tabular}
\end{table}

\subsubsection{RES\_KAR\_VENTIL}
Denne kommando er et svar der indeholder den tilstand ventilen er i.

\begin{table}[H]
\setlength{\parindent}{12pt}
\begin{tabular}{|l|lcc|}
\hline
\multicolumn{4}{|c|}{Kommado med argumenter}\\\hline
Kommando & 0x4 & & \\
Argumenter & STATE & & \\\hline
\end{tabular}
\end{table}

Svaret på en ventil forespørgsel er hvilken tilstand ventilen er blevet sat til.

\subsubsection{REQ\_KAR\_PUMPE}
Denne kommando er en forespørgsel på at styre karrets pumpe.

\begin{table}[H]
\setlength{\parindent}{12pt}
\begin{tabular}{|l|lcc|}
\hline
\multicolumn{4}{|c|}{Kommado med argumenter}\\\hline
Kommando & 0x5 & & \\
Argumenter & STATE & & \\\hline
\end{tabular}
\end{table}

Da pumpen kan køre med flere hastigheder sendes der en state der kan være følgende:

\begin{table}[H]
\setlength{\parindent}{12pt}
\begin{tabular}{|l|l|}
\hline
\multicolumn{2}{|c|}{Argumenter uddybet}\\\hline
Slukket & 0 \\
Lav hastighed & 25 \\
Middel hastighed & 50 \\
Høj hastighed & 75 \\\hline
\end{tabular}
\end{table}

Tallene betyder hvor mange procent duty cycle PWM signalet der styrer pumpen skal sættes til dvs, 50 er en duty cycle på 50\%.

\subsubsection{RES\_KAR\_PUMPE}
Denne kommando er et svar der indeholder den tilstand pumpen er i.

\begin{table}[H]
\setlength{\parindent}{12pt}
\begin{tabular}{|l|lcc|}
\hline
\multicolumn{4}{|c|}{Kommado med argumenter}\\\hline
Kommando & 0x6 & & \\
Argumenter & STATE & & \\\hline
\end{tabular}
\end{table}


Svaret på en pumpe forespørgsel er hvilken tilstand pumpen er blevet sat til. Altså om den er slukket eller kører med en af de tre hastigheder.


\subsubsection{REQ\_KAR\_OPRET}
Denne kommando er en forespørgsel give en sensorø der er koblet på et kar en adresse.

\begin{table}[H]
\setlength{\parindent}{12pt}
\begin{tabular}{|l|lcc|}
\hline
\multicolumn{4}{|c|}{Kommado med argumenter}\\\hline
Kommando & 0x7 & & \\
Argumenter & ADDR & & \\\hline
\end{tabular}
\end{table}


\subsubsection{RES\_KAR\_OPRET}
Denne kommando er et svar der fortæller om det lykkedes at oprette sensorøen på karret.

\begin{table}[H]
\setlength{\parindent}{12pt}
\begin{tabular}{|l|lcc|}
\hline
\multicolumn{4}{|c|}{Kommado med argumenter}\\\hline
Kommando & 0x8 & & \\
Argumenter & ADDR & & \\\hline
\end{tabular}
\end{table}


\subsubsection{REQ\_KAR\_OE\_LIST}
Denne kommando er en forespørgsel på at få af vide hvilke ø'er der er oprettede .

\begin{table}[H]
\setlength{\parindent}{12pt}
\begin{tabular}{|l|lcc|}
\hline
\multicolumn{4}{|c|}{Kommado med argumenter}\\\hline
Kommando & 0x9 & & \\
Argumenter & ingen & & \\\hline
\end{tabular}
\end{table}


\subsubsection{RES\_KAR\_OE\_LIST}
Denne kommando er et svar der fortæller hvilke ø'er der er koblet til karret. Der returneres 1 adresse per ø.

\begin{table}[H]
\setlength{\parindent}{12pt}
\begin{tabular}{|l|lcc|}
\hline
\multicolumn{4}{|c|}{Kommado med argumenter}\\\hline
Kommando & 0xA & & \\
Argumenter & ADDR & & \\\hline
\end{tabular}
\end{table}

\subsubsection{REQ\_KAR\_OE\_SENSOR\_DATA}
Denne kommando er en forespørgsel at få data fra de sensorer der er tilkoblet en sensorø. Argumentet er adressen på den ø der ønskes at der skal læses fra.

\begin{table}[H]
\setlength{\parindent}{12pt}
\begin{tabular}{|l|lcc|}
\hline
\multicolumn{4}{|c|}{Kommado med argumenter}\\\hline
Kommando & 0xB & & \\
Argumenter & ADDR & & \\\hline
\end{tabular}
\end{table}


\subsubsection{RES\_KAR\_OE\_SENSOR\_DATA}
Denne kommando er et svar der indeholder den data sensorø'en har opsamlet.

\begin{table}[H]
\setlength{\parindent}{12pt}
\begin{tabular}{|l|lccc|}
\hline
\multicolumn{5}{|c|}{Kommado med argumenter}\\\hline
Kommando & 0xC & & & \\
Argumenter & FS\_ADDR & FS\_STATUS & VALUE1 & VALUE2 \\\hline
\end{tabular}
\end{table}

Argumenterne beskrives her:

\begin{table}[H]
\setlength{\parindent}{12pt}
\begin{tabular}{|l|l|}
\hline
\multicolumn{2}{|c|}{Argumenter uddybet}\\\hline
FS\_ADDR 	& Dette er adressen på selve fieldsensoren \\
FS\_STATUS	& Status indikere om fieldsensoren er online (1) eller offline (0) \\
VALUE1 		& Heltals værdien af det sensoren udlæser \\
VALUE2 		& Hvis MSB er sat i denne indikere den at der er en halv der skal lægge til heltals værdien \\\hline
\end{tabular}
\end{table}

\subsubsection{REQ\_KAR\_OE\_VENTIL}
Denne kommando er en forespørgsel at åbne en ø ventil.

\begin{table}[H]
\setlength{\parindent}{12pt}
\begin{tabular}{|l|lcc|}
\hline
\multicolumn{4}{|c|}{Kommado med argumenter}\\\hline
Kommando & 0xD & & \\
Argumenter & OE\_ADDR & STATE & \\\hline
\end{tabular}
\end{table}

State er om ventilen skal være åben(1) eller lukket(0).

\subsubsection{RES\_KAR\_OE\_VENTIL}
Denne kommando er et svar der fortæller om det lykkedes at åbne ventilen det gøres ved at returnere den tilstand ventilen er i state har samme betydning som ovenfor.

\begin{table}[H]
\setlength{\parindent}{12pt}
\begin{tabular}{|l|lcc|}
\hline
\multicolumn{4}{|c|}{Kommado med argumenter}\\\hline
Kommando & 0xE & & \\
Argumenter & STATE & & \\\hline
\end{tabular}
\end{table}

State er om ventilen er åben(1) eller lukket(0).

\subsubsection{REQ\_KAR\_OE\_SENSOR\_TYPE}
Denne kommando er en forespørgsel på at få afvide hvilken type sensor der er koblet til en bestemt fieldsensor adresse.

\begin{table}[H]
\setlength{\parindent}{12pt}
\begin{tabular}{|l|lcc|}
\hline
\multicolumn{4}{|c|}{Kommado med argumenter}\\\hline
Kommando & 0xF & & \\
Argumenter & OE\_ADDR & FS\_ADDR & \\\hline
\end{tabular}
\end{table}


\subsubsection{RES\_KAR\_OE\_SENSOR\_TYPE}
Denne kommando er et svar der indeholder typen på den sensor der blev spurgt på.

\begin{table}[H]
\setlength{\parindent}{12pt}
\begin{tabular}{|l|lcc|}
\hline
\multicolumn{4}{|c|}{Kommado med argumenter}\\\hline
Kommando & 0x10 & & \\
Argumenter & TYPE & & \\\hline
\end{tabular}
\end{table}

Type kan være ??.

\subsection{Ø bus kommandoer}
Alle kommandoer er delt op på et "request" / "response" format dette er pga. af master/slave kommunikationen, det gør så at master altid kan forvente at få et svare og hvis ikke der bliver svaret er der sket en fejl.

Der er implementere følgende kommandoer på ØBussen:

\begin{table}[H]
\setlength{\parindent}{12pt}
\begin{tabular}{|l|l|}\hline
\multicolumn{2}{|c|}{ØBus\cellcolor[gray]{0.7}}\\\hline
Kommando & Beskrivelse \\\hline
REQ\_OE\_FS\_DATA 		& Forespørgsel på at få data fra sensor der er koblet til Sensor Ø \\\hline 
REQ\_OE\_FS\_DATA 		& Svar indeholdende sensor data								 \\\hline 
REQ\_OE\_VENTIL	   		& Forespørgsel på at skifte tilstand på ventil der er koblet til Sensor Ø \\\hline 
REQ\_OE\_VENTIL       		& Svar der indeholder den tilstand ventilen er sat til \\\hline 
\end{tabular}
\end{table}

\subsubsection{REQ\_OE\_FS\_DATA}
Denne kommando er en forespørgsel på data som Sensor Øen sensorer har indsamlet.

\begin{table}[H]
\setlength{\parindent}{12pt}
\begin{tabular}{|l|lcc|}
\hline
\multicolumn{4}{|c|}{Kommado med argumenter}\\\hline
Kommando & 0x1 & & \\
Argumenter & ingen & & \\\hline
\end{tabular}
\end{table}



\subsubsection{RES\_OE\_FS\_DATA}
Denne kommando er et svar der indeholder den data karrets sensorer har indsamlet.

\begin{table}[H]
\setlength{\parindent}{12pt}
\begin{tabular}{|l|lcccc|}
\hline
\multicolumn{6}{|c|}{Kommado med argumenter}\\\hline
Kommando & 0x2 & & & & \\
Argumenter & Addr & Type & Status & Value1 & Value2 \\\hline
\end{tabular}
\end{table}

Argumenterne gentages per Fieldsensor der er tilkoblet

\begin{table}[H]
\setlength{\parindent}{12pt}
\begin{tabular}{|l|l|}
\hline
\multicolumn{2}{|c|}{Argument uddybning}\\\hline
Addr & Adresse på Fieldsensoren. \\
Type & Type af Fieldsensoren. \\
Status & Status på Fieldsensoren 0x01 er aktiv og 0x00 svarer ikke. \\
Value1 & Er heltals værdien af den målte størrelse. \\
Value2 & Hvis MSB'en i denne er sat betyder det at der skal lægges en halv til value1.\\\hline
\end{tabular}
\end{table}

\subsubsection{REQ\_OE\_VENTIL}
Denne kommando er en forespørgsel på at styre Sensor Ø'ens ventiler.

\begin{table}[H]
\setlength{\parindent}{12pt}
\begin{tabular}{|l|lcc|}
\hline
\multicolumn{4}{|c|}{Kommado med argumenter}\\\hline
Kommando & 0x3 & & \\
Argumenter & ID & STATE & \\\hline
\end{tabular}
\end{table}

Her er ID den ventil der ønske manipuleret og STATE er hvilken tilstand der ønskes at ventilen skal stå i.\\\\
STATE kan være følgende:
\begin{table}[H]
\setlength{\parindent}{12pt}
\begin{tabular}{|l|l|}
\hline
\multicolumn{2}{|c|}{Argumenter uddybet}\\\hline
Lukket & 0x0 \\
Åben & 0x1 \\\hline
\end{tabular}
\end{table}

\subsubsection{RES\_OE\_VENTIL}
Denne kommando er et svar der indeholder den tilstand ventilen er i.

\begin{table}[H]
\setlength{\parindent}{12pt}
\begin{tabular}{|l|lcc|}
\hline
\multicolumn{4}{|c|}{Kommado med argumenter}\\\hline
Kommando & 0x4 & & \\
Argumenter & STATE & & \\\hline
\end{tabular}
\end{table}

Svaret på en ventil forespørgsel er hvilken tilstand ventilen er blevet sat til.


\subsection{GUI Protokol}

Til at kommunikere mellem GUI og CentralControl benyttes en TCP/IP forbindelse. Herunder er protokollen beskrevet. GUI har mulighed for at forbinde til CentralControl gennem en socket, og bruger dette til at starte eller stoppe handlinger, som skal udføres på CentralControl.\\\\

Data sendes i ren tekst (ASCII). Kommandoen efterfølges af argumenter adskilt med mellemrum. Kommandoer afsluttes med et carriage-return karakter efterfulgt af et newline karakter (\textbackslash r\textbackslash n).\\\\

Eksempel:\\
\textKode{MWSTART 5\textbackslash r\textbackslash n}\\

\begin{table}[H]
\setlength{\parindent}{12pt}
\begin{tabular}{|l|l|}\hline
\multicolumn{2}{|c|}{GUI Protokol\cellcolor[gray]{0.7}}\\\hline
Kommando & Beskrivelse \\\hline
MWSTART 		& Starter manuel vanding for et kar \\\hline 
MWSTOP  		& Stopper manuel vanding for et kar \\\hline 
IVALVEOPEN 		& Åbner indløbsventilen \\\hline 
IVALVECLOSE		& Lukker indløbsventilen \\\hline 
OVALVEOPEN 		& Åbner afløbsventilen \\\hline 
OVALVECLOSE		& Lukker afløbsventilen \\\hline 
\end{tabular}
\label{tab:GUIProtokol}
\caption{GUI Protokol kommandoer}
\end{table}


\subsubsection{MWSTART}
Starter manuel vanding for et kar.

\begin{table}[H]
\setlength{\parindent}{12pt}
\begin{tabular}{|l|lcc|}
\hline
\multicolumn{4}{|c|}{Kommado med argumenter}\\\hline
Kommando & MWSTART & & \\
Argumenter & KarID & & \\\hline
\end{tabular}
\end{table}


\subsubsection{MWSTOP}
Stopper manuel vanding for et kar.

\begin{table}[H]
\setlength{\parindent}{12pt}
\begin{tabular}{|l|lcc|}
\hline
\multicolumn{4}{|c|}{Kommado med argumenter}\\\hline
Kommando & MWSTOP & & \\
Argumenter & KarID & & \\\hline
\end{tabular}
\end{table}


\subsubsection{IVALVEOPEN}
Åbner indløbsventilen.

\begin{table}[H]
\setlength{\parindent}{12pt}
\begin{tabular}{|l|lcc|}
\hline
\multicolumn{4}{|c|}{Kommado med argumenter}\\\hline
Kommando & IVALVEOPEN & & \\
Argumenter & KarID & & \\\hline
\end{tabular}
\end{table}


\subsubsection{IVALVECLOSE}
Lukker indløbsventilen.

\begin{table}[H]
\setlength{\parindent}{12pt}
\begin{tabular}{|l|lcc|}
\hline
\multicolumn{4}{|c|}{Kommado med argumenter}\\\hline
Kommando & IVALVECLOSE & & \\
Argumenter & KarID & & \\\hline
\end{tabular}
\end{table}


\subsubsection{OVALVEOPEN}
Åbner afløbsventilen.

\begin{table}[H]
\setlength{\parindent}{12pt}
\begin{tabular}{|l|lcc|}
\hline
\multicolumn{4}{|c|}{Kommado med argumenter}\\\hline
Kommando & OVALVEOPEN & & \\
Argumenter & KarID & & \\\hline
\end{tabular}
\end{table}


\subsubsection{OVALVECLOSE}
Lukker afløbsventilen.

\begin{table}[H]
\setlength{\parindent}{12pt}
\begin{tabular}{|l|lcc|}
\hline
\multicolumn{4}{|c|}{Kommado med argumenter}\\\hline
Kommando & OVALVECLOSE & & \\
Argumenter & KarID & & \\\hline
\end{tabular}
\end{table}
